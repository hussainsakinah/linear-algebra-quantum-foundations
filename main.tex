\documentclass[12pt, a4paper, openany]{book}

%----------------------------------------------------------
% PACKAGES
%----------------------------------------------------------
\usepackage[utf8]{inputenc}
\usepackage[T1]{fontenc}
\usepackage{amsmath, amssymb, amsthm}
\usepackage{mathtools}
\usepackage{physics}          % bra-ket notation: \ket{}, \bra{}, \braket{}
\usepackage{geometry}
\usepackage{hyperref}
\usepackage{enumitem}
\usepackage{booktabs}
\usepackage{xcolor}
\usepackage{tcolorbox}
\usepackage{fancyhdr}
\usepackage{titlesec}
\usepackage{microtype}
\usepackage{lmodern}

\geometry{margin=2.5cm}

%----------------------------------------------------------
% THEOREM ENVIRONMENTS
%----------------------------------------------------------
\theoremstyle{definition}
\newtheorem{definition}{Definition}[section]
\newtheorem{example}{Example}[section]
\newtheorem{axiom}{Axiom}[section]

\theoremstyle{plain}
\newtheorem{theorem}{Theorem}[section]
\newtheorem{lemma}[theorem]{Lemma}
\newtheorem{proposition}[theorem]{Proposition}
\newtheorem{corollary}[theorem]{Corollary}

\theoremstyle{remark}
\newtheorem{remark}{Remark}[section]
\newtheorem{notation}{Notation}[section]

%----------------------------------------------------------
% CUSTOM COMMANDS
%----------------------------------------------------------
\newcommand{\N}{\mathbb{N}}
\newcommand{\Z}{\mathbb{Z}}
\newcommand{\Q}{\mathbb{Q}}
\newcommand{\R}{\mathbb{R}}
\newcommand{\C}{\mathbb{C}}
\newcommand{\F}{\mathbb{F}}
\newcommand{\inv}{^{-1}}
\newcommand{\conj}[1]{\overline{#1}}
\newcommand{\inner}[2]{\left\langle #1 \,\middle|\, #2 \right\rangle}
\newcommand{\outerp}[2]{\left| #1 \middle\rangle\!\middle\langle #2 \right|}
\newcommand{\spanset}{\operatorname{span}}

% Coloured box for key results
\tcbuselibrary{most}
\newtcolorbox{keyresult}[1][]{
  colback=blue!5!white,
  colframe=blue!60!black,
  fonttitle=\bfseries,
  title=Key Result,
  #1
}

%----------------------------------------------------------
% HEADER / FOOTER
%----------------------------------------------------------
\pagestyle{fancy}
\fancyhf{}
\fancyhead[LO]{\nouppercase{\rightmark}}
\fancyfoot[C]{\small\textit{Sakinah Faiza Hussain \textendash\ Mathematics for Quantum Computing}}
\renewcommand{\headrulewidth}{0.4pt}
\renewcommand{\footrulewidth}{0.4pt}

%----------------------------------------------------------
% HYPERREF SETUP
%----------------------------------------------------------
\hypersetup{
  colorlinks = true,
  linkcolor  = blue!60!black,
  citecolor  = green!50!black,
  urlcolor   = cyan!70!black,
  pdftitle   = {Mathematics for Quantum Computing},
  pdfauthor  = {},
}

%----------------------------------------------------------
% TITLE
%----------------------------------------------------------
\title{%
  \textbf{\Huge Mathematics for Quantum Computing}\\[0.5em]
  \large A Structured Reference from Lecture Notes
}

\author{Sakinah Faiza Hussain}
\date{\today}

%==========================================================
\begin{document}
%==========================================================

\maketitle

\tableofcontents
\clearpage

%----------------------------------------------------------
% CHAPTERS
%----------------------------------------------------------
\chapter{Algebraic Structures}

This chapter develops the hierarchy of algebraic structures — from the most
primitive (groupoids) to the richest (fields) — that underpin the mathematics
of quantum computing.  Every structure is a non-empty set equipped with one or
more binary operations satisfying progressively stronger axioms.

%==========================================================
\section{Binary Operations and Closure}
%==========================================================

\begin{definition}[Binary Operation]
  Let $R$ be a non-empty set.  A \emph{binary operation} $*$ on $R$ is a map
  \[
    * : R \times R \longrightarrow R, \qquad (a,b) \mapsto a * b.
  \]
  The operation is said to satisfy the \emph{closure property} if
  \[
    \forall\, a, b \in R \implies a * b \in R.
  \]
\end{definition}

\begin{remark}
  An algebraic structure $(R,\, *_1,\, *_2, \ldots)$ consists of a non-empty
  set $R$ together with one or more binary operations.  Closure is automatic
  by definition of a binary operation, but it is often stated explicitly for
  emphasis.
\end{remark}

\begin{example}
  Consider the integers $\Z$ under addition.  For any $x \in \Z$ we have
  $1 \cdot x = x \in \Z$, confirming closure.  The step-by-step justification
  that $2x + (y - x) = x + y$ uses commutativity, associativity, the
  distributive property, and the multiplicative identity:
  \begin{align*}
    2x + (y - x)
      &= 2x + (y + (-x))    && \text{(additive inverse notation)} \\
      &= 2x + ((-x) + y)    && \text{(commutativity of $+$)} \\
      &= (2x + (-x)) + y    && \text{(associativity of $+$)} \\
      &= (2x + (-1)x) + y   && \text{(notation: $-x = (-1)x$)} \\
      &= (2 + (-1))x + y    && \text{(distributive property)} \\
      &= 1x + y             && \text{(closure in $\Z$)} \\
      &= x + y.             && \text{(multiplicative identity)} \qed
  \end{align*}
\end{example}

%==========================================================
\section{The Algebraic Hierarchy}
%==========================================================

The following definitions build on one another; each adds one axiom to the
previous structure.

\begin{definition}[Groupoid]
  A non-empty set $R$ with a binary operation $*$ satisfying closure is called
  a \emph{groupoid}.
\end{definition}

\begin{definition}[Semigroup]
  A groupoid $(R, *)$ that additionally satisfies
  \begin{itemize}
    \item \textbf{Associativity:} $\forall\, a,b,c \in R,\quad (a*b)*c = a*(b*c)$
  \end{itemize}
  is called a \emph{semigroup}.
\end{definition}

\begin{definition}[Monoid]
  A semigroup $(R,*)$ that additionally contains an \emph{identity element}
  $e \in R$ satisfying
  \[
    a \mathbin{\square} e = e \mathbin{\square} a = a \quad \forall\, a \in R
  \]
  is called a \emph{monoid}.
\end{definition}

\begin{definition}[Group]
  A monoid $(R,*)$ in which every element has an \emph{inverse}, i.e.,
  \[
    \forall\, a \in R\; \exists\, a\inv \in R \text{ such that }
    a * a\inv = a\inv * a = e,
  \]
  is called a \emph{group}.
\end{definition}

\begin{definition}[Abelian (Commutative) Group]
  A group $(R,*)$ satisfying
  \begin{itemize}
    \item \textbf{Commutativity:} $\forall\, a,b \in R,\quad a*b = b*a$
  \end{itemize}
  is called an \emph{Abelian group}.
\end{definition}

%==========================================================
\section{Rings}
%==========================================================

\begin{definition}[Ring]
  A triple $(R, +, \cdot)$ is a \emph{ring} if $(R,+)$ is an Abelian group,
  multiplication $\cdot$ is associative and closed, and the distributive laws
  hold:
  \begin{align*}
    (a + b)\cdot c &= a\cdot c + b\cdot c, \\
    a \cdot (b + c) &= a\cdot b + a\cdot c.
  \end{align*}
  Concretely, the twelve axioms are:
  \begin{enumerate}
    \item \textbf{Closure under $+$:} $a,b\in R \Rightarrow a+b\in R$.
    \item \textbf{Associativity of $+$:} $(a+b)+c = a+(b+c)$.
    \item \textbf{Additive identity:} $\exists\, 0\in R$ s.t.\ $a+0=0+a=a$.
    \item \textbf{Additive inverse:} $\forall\, a\in R\;\exists\, {-a}\in R$ s.t.\ $a+(-a)=0$.
    \item \textbf{Commutativity of $+$:} $a+b=b+a$.
    \item \textbf{Closure under $\cdot$:} $a,b\in R \Rightarrow a\cdot b\in R$.
    \item \textbf{Associativity of $\cdot$:} $(a\cdot b)\cdot c = a\cdot(b\cdot c)$.
    \item \textbf{Multiplicative identity} (if present): $\exists\, 1\in R$ s.t.\ $a\cdot 1=1\cdot a=a$.
    \item \textbf{Multiplicative inverse} (if present): $\forall\, a\neq 0\;\exists\, a\inv$ s.t.\ $a\cdot a\inv=1$.
    \item \textbf{Commutativity of $\cdot$} (if present): $a\cdot b = b\cdot a$.
    \item \textbf{Distributivity:} $(a+b)c = ac+bc$ and $a(b+c)=ab+ac$.
    \item \textbf{Non-triviality:} $0 \neq 1$ (excludes the trivial ring; also forces uniqueness of identity and inverses).
  \end{enumerate}
\end{definition}

\begin{definition}[Ring with Unity]
  A ring satisfying axiom~8 (multiplicative identity) is called a
  \emph{ring with unity}.
\end{definition}

\begin{definition}[Commutative Ring with Unity]
  A ring with unity that also satisfies axiom~10 (commutativity of $\cdot$)
  is called a \emph{commutative ring with unity}.
\end{definition}

%==========================================================
\section{Fields}
%==========================================================

\begin{definition}[Field]
  A \emph{field} $(F, +, \cdot)$ is a set $F$ with two binary operations
  satisfying all twelve ring axioms (including multiplicative identity,
  multiplicative inverses for non-zero elements, and commutativity of
  multiplication).  Equivalently:
  \[
    \text{Field} = \text{Abelian group under } {+}
                   \;\cup\; \text{Abelian group under } {\cdot}
                   \;\cup\; \text{Distributivity}
                   \;\cup\; (0 \neq 1).
  \]
\end{definition}

\begin{example}
  The classical fields are:
  \[
    \R \;(\mathbb{R}\setminus\{0\}),\quad
    \C \;(\mathbb{C}\setminus\{0\}),\quad
    \Q \;(\mathbb{Q}\setminus\{0\}).
  \]
\end{example}

\subsection{Field Extensions}

\begin{definition}[Field Extension]
  Let $F$ and $K$ be fields with $F \subseteq K$.  Then $K$ is an
  \emph{extension field} of $F$, written $K/F$.
\end{definition}

\begin{example}[$\Q(\alpha)$]
  Since $\sqrt{2} \notin \Q$, the smallest field containing both $\Q$ and
  $\sqrt{2}$ is
  \[
    \Q(\sqrt{2}) = \{\, a + b\sqrt{2} : a,b \in \Q \,\}.
  \]
  This is an \emph{algebraic extension} of $\Q$ and forms an Abelian group
  under both $+$ and $\cdot$, satisfies distributivity, and has $0 \neq 1$.
\end{example}

\begin{definition}[Extended Field $F(x)$]
  Given a field $F$, the field of \emph{rational functions} over $F$ is
  \[
    F(x) = \left\{\, \frac{f(x)}{g(x)} \;\middle|\;
            f, g \text{ polynomials with coefficients in } F,\; g \neq 0 \,\right\}.
  \]
  In particular, $\R(i) = \C$ (extended reals via the imaginary unit
  $i = \sqrt{-1}$) and $\C(x)$ consists of rational functions with complex
  coefficients, e.g.\ $\dfrac{ix+3}{3-ix^3}$.
\end{definition}

\begin{remark}
  There are infinitely many extension fields of $\Q$, one for each algebraic
  (or transcendental) element adjoined.
\end{remark}

\chapter{Linear Transformations and Systems of Equations}

%==========================================================
\section{Systems of Linear Equations}
%==========================================================

A system of linear equations can have three qualitatively different solution
sets, visualised geometrically as intersecting lines in $\R^2$:

\begin{enumerate}
  \item \textbf{Unique solution} — lines intersect at exactly one point.
  \item \textbf{No solution} — lines are parallel and distinct.
  \item \textbf{Infinitely many solutions} — lines coincide.
\end{enumerate}

\begin{example}
  \[
    \begin{cases} x + y = 6 \\ 2x - y = 3 \end{cases}
    \xrightarrow{\text{subtract}}
    -x = 3 \implies x = -3,\; y = 9.
  \]
  Unique solution. \qed
\end{example}

%==========================================================
\section{Matrix Notation (Cayley)}
%==========================================================

A system of $m$ equations in $n$ unknowns is written compactly as
\[
  A\mathbf{x} = \mathbf{b},
\]
where $A$ is an $m\times n$ matrix, $\mathbf{x}$ is the $n\times 1$ column of
unknowns, and $\mathbf{b}$ is the $m\times 1$ right-hand-side vector.

\begin{example}
  \[
    \begin{pmatrix} 1 & 1 \\ 2 & -1 \end{pmatrix}
    \begin{pmatrix} x \\ y \end{pmatrix}
    = \begin{pmatrix} 6 \\ 4 \end{pmatrix},
    \qquad
    \begin{pmatrix} 1 & 2 & 1 \\ 1 & -3 & 1 \\ 2 & -1 & 1 \end{pmatrix}
    \begin{pmatrix} x \\ y \\ z \end{pmatrix}
    = \begin{pmatrix} 0 \\ 3 \\ 5 \end{pmatrix}.
  \]
  The matrix is \emph{responsible for the transformation} of the solution
  space.
\end{example}

%==========================================================
\section{Linear Transformations}
%==========================================================

\begin{definition}[Linear Transformation]
  Let $V$ and $W$ be vector spaces over the same field $F$.  A map
  $T : V \to W$ is a \emph{linear transformation} if it preserves the vector
  space structure:
  \begin{align}
    T(\mathbf{u} + \mathbf{v}) &= T(\mathbf{u}) + T(\mathbf{v})
      && \forall\, \mathbf{u},\mathbf{v}\in V, \label{eq:lt-add}\\
    T(\gamma\, \mathbf{u}) &= \gamma\, T(\mathbf{u})
      && \forall\, \mathbf{u}\in V,\; \gamma \in F. \label{eq:lt-scalar}
  \end{align}
\end{definition}

\begin{example}[Geometric transformations on $\R^2$]
  Define
  \[
    T(x,y) = (x, y), \qquad T(x,y) = (x+y,\; 2x-y).
  \]
  The second map sends integer-lattice points to a \emph{tilted} lattice —
  a square grid becomes a tilted rectangle.  This geometric distortion is the
  hallmark of a non-trivial linear transformation.
\end{example}

\subsection{Reflections as Linear Transformations}

\begin{example}
  Define the two maps $G,H: \R^2 \to \R^2$ by
  \[
    G(x,y) = (x,\,-y), \qquad H(x,y) = (y,\,-x).
  \]
  Their matrix representations are
  \[
    G = \begin{pmatrix} 1 & 0 \\ 0 & -1 \end{pmatrix}, \qquad
    H = \begin{pmatrix} 0 & -1 \\ 1 & 0 \end{pmatrix}.
  \]
  The composition $GH$ satisfies
  \begin{align*}
    (GH)(x,y)
      &= G(H(x,y)) = G(y,-x) = (y,x),
  \end{align*}
  so
  \[
    GH = \begin{pmatrix} 1 & 0 \\ 0 & -1 \end{pmatrix}
         \begin{pmatrix} 0 & -1 \\ 1 & 0 \end{pmatrix}
       = \begin{pmatrix} 0 & 1 \\ 1 & 0 \end{pmatrix},
    \qquad
    \begin{pmatrix} 0 & 1 \\ 1 & 0 \end{pmatrix}
    \begin{pmatrix} x \\ y \end{pmatrix}
    = \begin{pmatrix} y \\ x \end{pmatrix}.
  \]
  \textbf{Verification:} $GH(1,2)$: first $H(1,2)=(2,-1)$, then $G(2,-1)=(2,1)$.
  \qed
\end{example}

%==========================================================
\section{Composition of Linear Transformations as Matrix Multiplication}
%==========================================================

\begin{proposition}[Matrix multiplication encodes composition]
  If $T_1 : \R^2 \to \R^2$ has matrix $\begin{psmallmatrix} a & b \\ c & d \end{psmallmatrix}$
  and $T_2$ has matrix $\begin{psmallmatrix} A & B \\ C & D \end{psmallmatrix}$,
  then $(T_2 \circ T_1)$ has matrix
  \[
    \begin{pmatrix} A & B \\ C & D \end{pmatrix}
    \begin{pmatrix} a & b \\ c & d \end{pmatrix}
    = \begin{pmatrix} Aa+Bc & Ab+Bd \\ Ca+Dc & Cb+Dd \end{pmatrix}.
  \]
\end{proposition}

\begin{proof}
  Let $x' = Ax + By$, $y' = Cx + Dy$ and $x = ax_0 + by_0$, $y = cx_0 + dy_0$.
  Substituting:
  \begin{align*}
    x' &= A(ax_0+by_0) + B(cx_0+dy_0) = (Aa+Bc)x_0 + (Ab+Bd)y_0, \\
    y' &= C(ax_0+by_0) + D(cx_0+dy_0) = (Ca+Dc)x_0 + (Cb+Dd)y_0. \qed
  \end{align*}
\end{proof}

%==========================================================
\section{Abstract Formalism: Linear Transformations on $\R^2$}
%==========================================================

Let $(a,b),(c,d) \in \R^2$ (a 2-dimensional plane, i.e.\ a \emph{space}).
The scalar field $\gamma \in \R$ (or $\C$) specifies the number of components.
A map $T : \R^2 \to \R^2$ is linear if and only if

\begin{equation}\label{eq:lin-formal}
  T\bigl[(a,b)+(c,d)\bigr] = T(a,b)+T(c,d), \qquad
  T\bigl(\gamma(a,b)\bigr) = \gamma\, T(a,b).
\end{equation}

\begin{remark}
  A vector space with \emph{infinitely many dimensions} is called a
  \textbf{Hilbert Space} (see Chapter~\ref{chap:hilbert}).
\end{remark}

\chapter{Complex Numbers}

Complex numbers are the native number system of quantum mechanics.  Every
quantum amplitude is complex, and the modulus-squared gives a probability.

%==========================================================
\section{Definition and the Argand Plane}
%==========================================================

\begin{definition}[Imaginary Unit]
  The \emph{imaginary unit} is the number $i$ satisfying $i^2 = -1$, i.e.\
  $i = \sqrt{-1}$.
\end{definition}

\begin{definition}[Complex Number]
  A \emph{complex number} is an expression of the form
  \[
    z = x + iy, \qquad x,y \in \R,
  \]
  where $x = \operatorname{Re}(z)$ is the \emph{real part} and
  $y = \operatorname{Im}(z)$ is the \emph{imaginary part}.
\end{definition}

The set $\C$ of all complex numbers is visualised on the \textbf{Argand plane}
(complex plane), with the real axis horizontal and the imaginary axis vertical.

%==========================================================
\section{Polar Form and Euler's Formula}
%==========================================================

\begin{definition}[Polar Form]
  Any $z = x+iy \in \C$ can be written in \emph{polar form}
  \[
    z = r(\cos\theta + i\sin\theta),
  \]
  where $r = |z| = \sqrt{x^2+y^2}$ is the \emph{modulus} and
  $\theta = \arg(z)$ is the \emph{argument}.
\end{definition}

\begin{theorem}[Euler's Formula]
  \[
    e^{i\theta} = \cos\theta + i\sin\theta.
  \]
  Hence the polar form simplifies to $z = r e^{i\theta}$.
\end{theorem}

\begin{corollary}[Euler's Identity]
  Setting $\theta = \pi$:
  \[
    \boxed{e^{i\pi} + 1 = 0.}
  \]
\end{corollary}

\begin{proposition}[Product of complex numbers]
  If $z_1 = r_1 e^{i\theta_1}$ and $z_2 = r_2 e^{i\theta_2}$, then
  \[
    z_1 z_2 = r_1 r_2\, e^{i(\theta_1+\theta_2)}.
  \]
  Multiplication \emph{adds arguments} and \emph{multiplies moduli}.
\end{proposition}

%==========================================================
\section{Complex Conjugate}
%==========================================================

\begin{definition}[Complex Conjugate]
  The \emph{complex conjugate} of $z = x+iy$ is $z^* = x - iy$.
  Geometrically, $z^*$ is the reflection of $z$ through the real axis
  (angle $-\theta$).
\end{definition}

\begin{proposition}[Modulus via conjugate]
  \[
    z z^* = (x+iy)(x-iy) = x^2 + y^2 = |z|^2.
  \]
  Hence $\abs{z} = \sqrt{z z^*}$.
\end{proposition}

\begin{proof}
  Using Euler's form:
  \[
    z z^* = (re^{i\theta})(re^{-i\theta}) = r^2 e^0 = r^2. \qed
  \]
\end{proof}

\begin{remark}[Dual number system]
  The pair $(z, z^*)$ forms a \emph{dual number system}: for every $z \in \C$
  there exists a unique $z^*$, so that $\forall z\; \exists!\, z^*$.
\end{remark}

%==========================================================
\section{Phase Factor}
%==========================================================

\begin{definition}[Phase Factor]
  A \emph{phase factor} is a complex number of unit modulus:
  \[
    z = e^{i\theta} = \cos\theta + i\sin\theta, \qquad |z| = 1.
  \]
  Phase factors are ubiquitous in quantum mechanics: quantum states often
  differ only by a phase, and this phase carries physical information (e.g.\
  interference).
\end{definition}

\begin{remark}
  In quantum calculations we work freely in $\C$, but the \emph{final answer
  must be real} — probabilities and expectation values of Hermitian observables
  are always real numbers.
\end{remark}

%==========================================================
\section{Complex Vector Space}
%==========================================================

\begin{definition}[Complex Vector Space]
  A \emph{complex vector space} $V(\C)$ is a vector space over the field
  $\C$ of scalars.  This is the setting for quantum mechanics, where we take
  $F = \C$.
\end{definition}

The complex conjugate of the scalar field $\C$ is $\C^*$; the dual vector
space $V^*(\C^*)$ is the \emph{complex conjugate vector space}.

\chapter{Vector Spaces}

%==========================================================
\section{What is ``Space'' in Mathematics?}
%==========================================================

In everyday language, \emph{space} is physical.  In mathematics, a
\emph{space} is a non-empty set equipped with a \textbf{mathematical
structure} — a collection of axioms, theorems, and definitions that constrain
the set's behaviour.

\begin{example}
  $A = \{1, 2, 3\}$ is an \emph{ordinary} (simple) set with no additional
  structure.  By contrast, $(\R, +)$ follows the axioms of an Abelian group
  and is therefore a \emph{space}.
\end{example}

%==========================================================
\section{Definition of a Vector Space}
%==========================================================

\begin{definition}[Vector Space]\label{def:vector-space}
  A \emph{vector space} over a field $F$ is a non-empty set $V$ together with
  two operations:
  \begin{itemize}
    \item \textbf{Vector addition:} $+: V\times V \to V$,
    \item \textbf{Scalar multiplication:} $\cdot: F\times V \to V$,
  \end{itemize}
  satisfying the following nine axioms for all $\mathbf{u},\mathbf{v},\mathbf{w}\in V$
  and $a,b\in F$:

  \begin{enumerate}[label=\textbf{VS\arabic*.}]
    \item $\mathbf{u}+\mathbf{v} = \mathbf{v}+\mathbf{u}$
          \hfill(commutativity of $+$)
    \item $(\mathbf{u}+\mathbf{v})+\mathbf{w} = \mathbf{u}+(\mathbf{v}+\mathbf{w})$
          \hfill(associativity of $+$)
    \item $\exists\, \mathbf{0}\in V$ such that $\mathbf{0}+\mathbf{u}=\mathbf{u}+\mathbf{0}=\mathbf{u}$
          \hfill(additive identity)
    \item $\forall\,\mathbf{u}\in V\;\exists\,(-\mathbf{u})\in V$ such that
          $\mathbf{u}+(-\mathbf{u})=\mathbf{0}$
          \hfill(additive inverse)
    \item $1\cdot\mathbf{u} = \mathbf{u}$ \hfill(scalar identity)
    \item $a\cdot(b\cdot\mathbf{u}) = (ab)\cdot\mathbf{u}$
          \hfill(compatibility of scalar multiplication)
    \item $(a+b)\mathbf{u} = a\mathbf{u} + b\mathbf{u}$
          \hfill(distributivity over scalar addition)
    \item $a(\mathbf{u}+\mathbf{v}) = a\mathbf{u} + a\mathbf{v}$
          \hfill(distributivity over vector addition)
    \item $\exists\,\mathbf{0}\in V$ such that $\mathbf{u}\cdot\mathbf{0}=\mathbf{0}$
          \hfill(zero vector)
  \end{enumerate}
\end{definition}

\begin{remark}
  Axioms VS1--VS4 make $(V,+)$ an Abelian group.  The scalar multiplication
  axioms VS5--VS8 couple the field structure to the group structure.
\end{remark}

%==========================================================
\section{Examples of Vector Spaces}
%==========================================================

\begin{example}
  The following are all vector spaces over appropriate fields:
  \begin{enumerate}
    \item \textbf{Polynomials of degree $\leq 1$:}
          $u = x+1$, $v = 2x-3$; then $u+v = 3x-2$ and $5u = 5x+5$,
          both first-degree polynomials. \checkmark
    \item \textbf{$2\times 2$ matrices:}
          $u = \begin{psmallmatrix}2&1\\1&3\end{psmallmatrix}$,
          $v = \begin{psmallmatrix}-1&0\\1&1\end{psmallmatrix}$;
          addition and scalar multiplication are component-wise. \checkmark
    \item \textbf{Indefinite integrals:}
          $u = \int f(x)\,dx$, $v = \int g(x)\,dx$;
          $u+v = \int(f+g)(x)\,dx$. \checkmark
    \item \textbf{Derivatives:}
          $u = \tfrac{d}{dx}f(x)$, $v = \tfrac{d}{dx}g(x)$. \checkmark
    \item \textbf{Real scalars:} $u = 1$, $v = 2$. \checkmark
  \end{enumerate}
\end{example}

%==========================================================
\section{Subspaces}
%==========================================================

\begin{definition}[Subspace]
  Let $V(F)$ be a vector space.  A non-empty subset $W \subseteq V$ is a
  \emph{subspace} of $V$ if $W$ is itself a vector space over $F$ under the
  same operations.  Equivalently, $W$ is a subspace if and only if:
  \begin{enumerate}
    \item $\mathbf{0} \in W$,
    \item $\mathbf{u},\mathbf{v}\in W \Rightarrow \mathbf{u}+\mathbf{v}\in W$,
    \item $a\in F,\;\mathbf{u}\in W \Rightarrow a\mathbf{u}\in W$.
  \end{enumerate}
\end{definition}

%==========================================================
\section{Metric Spaces}
%==========================================================

\begin{definition}[Metric Space]
  Let $X$ be a non-empty set.  A function $d: X\times X \to \R$ is a
  \emph{metric} (distance function) on $X$ if, for all $x,y,z\in X$:
  \begin{enumerate}[label=(\roman*)]
    \item $d(x,y) \geq 0$ and $d(x,y)=0 \iff x=y$,
    \item $d(x,y) = d(y,x)$ \hfill(symmetry),
    \item $d(x,z) \leq d(x,y) + d(y,z)$ \hfill(triangle inequality).
  \end{enumerate}
  The pair $(X,d)$ is then called a \emph{metric space}.
\end{definition}

\begin{example}
  On $X = \R$, the function $d(x,y) = |x-y|$ is a metric (the standard
  Euclidean distance).  Any $x$ and $y$ satisfying the three conditions above
  define a metric space — they need not be numbers.
\end{example}

%==========================================================
\section{Key Group-Theoretic Theorems}
%==========================================================

\begin{theorem}[Uniqueness of Identity]
  If $(G,*)$ is a group, then the identity element $e$ is unique.
\end{theorem}

\begin{proof}
  Suppose $e$ and $e'$ are both identities.  Then
  $e = e * e' = e'$, since $e'$ is an identity for $e$ and $e$ is an
  identity for $e'$. \qed
\end{proof}

\begin{theorem}[Uniqueness of Inverses]
  In a group $(G,*)$, every element has a unique inverse.
\end{theorem}

\begin{proof}
  Let $b$ and $c$ both be inverses of $a$.  Then
  $b = b*e = b*(a*c) = (b*a)*c = e*c = c$. \qed
\end{proof}

%==========================================================
\section{Internal and External Compositions}
%==========================================================

\begin{definition}[Internal Composition]
  A composition $*$ on a non-empty set $V$ is \emph{internal} if
  \[
    \forall\, \alpha,\beta \in V \implies \alpha * \beta \in V.
  \]
  The result is uniquely determined.
\end{definition}

\begin{definition}[External Composition]
  Let $F$ be a field (of scalars) and $V$ a non-empty set.  A composition
  $\circ: F \times V \to V$ is \emph{external} if
  \[
    \forall\, a \in F,\; \alpha \in V \implies a \circ \alpha \in V,
  \]
  and the result is unique.
\end{definition}

\begin{remark}
  Vector addition is an \emph{internal} composition on $V$; scalar
  multiplication is an \emph{external} composition from $F$ into $V$.  Both
  are required to define a vector space.
\end{remark}

\chapter{Normed Spaces and Inner Product Spaces}

The progression \textbf{Vector Space} $\subset$ \textbf{Normed Space}
$\subset$ \textbf{Inner Product Space} $\subset$ \textbf{Hilbert Space}
adds richer geometric structure at each step.

%==========================================================
\section{Normed Spaces}
%==========================================================

A vector space has no built-in notion of length or distance.  A norm supplies
both.

\begin{definition}[Norm]
  Let $V$ be a vector space over $F \in \{\R, \C\}$.  A \emph{norm} on $V$
  is a function $\norm{\cdot}: V \to \R_{\geq 0}$ satisfying, for all
  $\mathbf{u},\mathbf{v}\in V$ and $a\in F$:
  \begin{enumerate}[label=\textbf{N\arabic*.}]
    \item \textbf{Non-negativity:} $\norm{\mathbf{u}} \geq 0$, and
          $\norm{\mathbf{u}}=0 \iff \mathbf{u}=\mathbf{0}$.
    \item \textbf{Homogeneity:} $\norm{a\mathbf{u}} = |a|\,\norm{\mathbf{u}}$.
    \item \textbf{Triangle inequality:} $\norm{\mathbf{u}+\mathbf{v}}
          \leq \norm{\mathbf{u}} + \norm{\mathbf{v}}$.
  \end{enumerate}
  The pair $(V, \norm{\cdot})$ is a \emph{normed space}.
\end{definition}

\begin{remark}
  The norm assigns a non-negative real number to each vector (its
  ``length''), but without an inherent notion of actual physical distance or
  angle between vectors.  The norm also induces a metric:
  \[
    d(\mathbf{u},\mathbf{v}) \coloneqq \norm{\mathbf{u}-\mathbf{v}}.
  \]
\end{remark}

%==========================================================
\section{Inner Product Spaces}
%==========================================================

\begin{definition}[Inner Product]\label{def:inner-product}
  Let $V$ be a vector space over $F$.  An \emph{inner product} is a map
  $\langle\cdot,\cdot\rangle: V\times V \to F$ (producing only
  non-negative real outputs when both arguments coincide) satisfying:
  \begin{enumerate}[label=\textbf{IP\arabic*.}]
    \item \textbf{Linearity in the first argument:}
          $\langle ax_1 + bx_2,\, y \rangle = a\langle x_1, y\rangle
           + b\langle x_2, y\rangle$.
    \item \textbf{Conjugate symmetry:}
          $\langle x, y\rangle = \overline{\langle y, x\rangle}$.
    \item \textbf{Positive definiteness:}
          $\langle x,x\rangle \geq 0$, and $\langle x,x\rangle = 0 \iff x = 0$.
  \end{enumerate}
  The pair $(V,\langle\cdot,\cdot\rangle)$ is an \emph{inner product space}
  (IPS).
\end{definition}

\begin{remark}
  Geometrically, the inner product $\langle x,y\rangle$ encodes how much $x$
  ``points in the direction of'' $y$ and how their magnitudes contribute to
  their alignment.  In physics it is often written
  $\vec{A}\cdot\vec{B} = |\vec{A}||\vec{B}|\cos\theta$.
\end{remark}

\subsection{Norm Induced by the Inner Product}

\begin{proposition}
  Every inner product induces a norm by
  \[
    \norm{x} \coloneqq \sqrt{\langle x, x\rangle}.
  \]
\end{proposition}

\begin{example}
  For $\vec{A}$ pointing along itself,
  $\vec{A}\cdot\vec{A} = |\vec{A}|^2\cos 0 = |\vec{A}|^2$,
  so $|\vec{A}| = \sqrt{\vec{A}\cdot\vec{A}}$.
\end{example}

%==========================================================
\section{Orthogonality, Normalization, and Orthonormality}
%==========================================================

\begin{definition}[Orthogonality]
  Two vectors $\ket{A}$ and $\ket{B}$ in an IPS are \emph{orthogonal} if
  $\braket{A}{B} = 0$.  Orthogonal vectors are linearly independent.
\end{definition}

\begin{definition}[Normalization]
  A vector $\ket{A}$ is \emph{normalized} (a unit vector) if
  $\braket{A}{A} = 1$.
\end{definition}

\begin{definition}[Orthonormality]
  A set of vectors $\{\ket{A}, \ket{B}, \ldots\}$ is \emph{orthonormal} if
  \[
    \braket{A}{B} = \delta_{AB} =
    \begin{cases} 0 & A \neq B, \\ 1 & A = B. \end{cases}
  \]
\end{definition}

%==========================================================
\section{Topological Properties of Spaces}
%==========================================================

\begin{definition}[Topological Space]
  A \emph{topological space} is the most general type of space: a set $X$
  together with a \emph{topology} (a collection $\mathcal{T}$ of
  ``open'' subsets of $X$) satisfying:
  \begin{enumerate}
    \item The union of \emph{any} collection of open sets is open.
    \item The intersection of a \emph{finite} number of open sets is open.
  \end{enumerate}
  Examples: function spaces, polynomial spaces.
\end{definition}

Three key properties of topological spaces relevant to quantum mechanics:

\begin{definition}[Connectedness]
  A space is \emph{connected} if it cannot be split into two
  non-overlapping, disjoint, non-empty open sets.
\end{definition}

\begin{definition}[Continuity]
  A function $f$ between topological spaces is \emph{continuous} (smooth) if
  small changes in the input produce only small changes in the output.
\end{definition}

\begin{definition}[Compactness]
  A space is \emph{compact} if it is both bounded and closed, making it
  well-behaved for convergence and containment.  For example, the open
  interval $(a,b)$ is \emph{not} compact; the closed interval $[a,b]$
  \emph{is} compact.
\end{definition}

%==========================================================
\section{Unit Vectors and Normalization}
%==========================================================

\begin{definition}[Unit Vector]
  A vector $\ket{\psi}$ with $\norm{\ket{\psi}} = 1$ is a \emph{unit vector}.
\end{definition}

\begin{proposition}[Normalization]
  Any non-zero vector $\ket{\psi}$ can be converted to a unit vector by
  dividing by its norm:
  \[
    \widehat{\psi} = \frac{1}{\norm{\ket{\psi}}}\ket{\psi}.
  \]
\end{proposition}

\begin{example}
  Let $\ket{\psi} = \begin{pmatrix}3\\1\end{pmatrix}$.
  Then $\norm{\ket{\psi}} = \sqrt{9+1} = \sqrt{10}$, and
  \[
    \text{unit vector} = \frac{1}{\sqrt{10}}\begin{pmatrix}3\\1\end{pmatrix}
    = \begin{pmatrix}3/\sqrt{10}\\1/\sqrt{10}\end{pmatrix},
    \qquad
    \left\|\begin{pmatrix}3/\sqrt{10}\\1/\sqrt{10}\end{pmatrix}\right\|
    = \sqrt{\tfrac{9}{10}+\tfrac{1}{10}} = 1. \quad\checkmark
  \]
\end{example}

\chapter{Hilbert Spaces}
\label{chap:hilbert}

A \emph{Hilbert space} is the mathematical arena of quantum mechanics.  It
combines the algebraic richness of an inner product space with the analytical
property of completeness.

%==========================================================
\section{Cauchy Sequences and Convergence}
%==========================================================

\begin{definition}[Cauchy Sequence]
  A sequence $\{x_n\}$ in a metric space $(X,d)$ (or normed space) is a
  \emph{Cauchy sequence} if
  \[
    \forall\, \varepsilon > 0\;\; \exists\, N \in \N \;\text{ such that }\;
    \forall\, m,n \geq N: \quad |x_m - x_n| < \varepsilon.
  \]
  Intuitively, consecutive terms cluster arbitrarily tightly as $n\to\infty$.
\end{definition}

\begin{example}
  $\left\{\dfrac{1}{n}\right\}_{n=1}^{\infty} = 1,\tfrac{1}{2},\tfrac{1}{3},\ldots$
  is Cauchy: for any $\varepsilon>0$ choose $N > 2/\varepsilon$, then for
  $m,n\geq N$,
  \[
    |x_m-x_n| \leq |x_m| + |x_n| < \tfrac{2}{N} < \varepsilon.
  \]
\end{example}

\begin{definition}[Convergent Sequence]
  A sequence $\{x_n\}$ \emph{converges} to a limit $L$ if
  \[
    \lim_{n\to\infty} x_n = L, \quad\text{i.e.,}\quad
    \forall\,\varepsilon>0\;\exists\,N:\; n\geq N \Rightarrow |x_n - L| < \varepsilon.
  \]
  Every convergent sequence is Cauchy, but the converse requires completeness.
\end{definition}

\begin{definition}[Dart-board analogy]
  Think of convergence as a dart game: every throw lands strictly closer to
  the bull's-eye than the previous throw.  The sequence of positions is Cauchy;
  the bull's-eye is the limit.
\end{definition}

%==========================================================
\section{Completeness}
%==========================================================

\begin{definition}[Complete Normed Space (Banach Space)]
  A normed space $X$ is \emph{complete} if every Cauchy sequence in $X$
  has a limit \emph{within $X$}.  A complete normed space is called a
  \textbf{Banach space}.
\end{definition}

\begin{example}[Incompleteness of $\Q$]
  Consider the sequence in $\Q$ defined by $x_n = \sum_{k=0}^{n}\dfrac{1}{k!}$.
  It is Cauchy, but its limit $e = 2.71828\ldots$ is irrational, hence
  $e \notin \Q$.  Therefore $(\Q, |\cdot|)$ is \emph{not} complete.
\end{example}

\begin{example}[Motivation from physics]
  The sequence of hydrogen energy levels $E_n = -13.6\,\text{eV}/n^2$.
  As $n\to\infty$, $E_n \to 0$.  Completeness requires that $0$ be in the
  same space as the $E_n$, which it is (the continuum of unbound states).
\end{example}

%==========================================================
\section{Hilbert Space}
%==========================================================

\begin{definition}[Hilbert Space]
  A \textbf{Hilbert space} $\mathcal{H}$ is a \emph{complete inner product
  space} (a complete IPS):
  \[
    \mathcal{H} = \underbrace{\text{IPS}}_{\text{properties}}
                  + \underbrace{\text{completeness}}_{\text{every Cauchy seq.\ converges in }\mathcal{H}}.
  \]
\end{definition}

\subsection{Quantum Mechanical Interpretation}

\begin{enumerate}
  \item \textbf{States.} The state of a quantum particle is represented by a
        vector (ket) $\ket{\psi} \in \mathcal{H}$.
  \item \textbf{Infinite dimension.} $\mathcal{H}$ is generically
        infinite-dimensional, encoding the infinite complexity of states
        available to a particle.  A particle's state is a superposition of
        infinitely many basis states.
  \item \textbf{Finite vs.\ infinite.} In quantum \emph{computing} (qubits,
        spin), finite-dimensional Hilbert spaces suffice (2 or 3 dimensions).
        In full quantum \emph{mechanics} (position, momentum, energy), we
        need infinite-dimensional $\mathcal{H}$.
  \item \textbf{Many particles.} For a system of many particles, the total
        Hilbert space is the tensor product of individual Hilbert spaces:
        \[
          \mathcal{H}_{\text{total}} = \mathcal{H}_1 \otimes \mathcal{H}_2
                                        \otimes \cdots
        \]
        The space grows very large, but each individual factor remains
        infinite-dimensional.
  \item \textbf{Physical examples.}
        \begin{itemize}
          \item Spin of an electron, polarisation of a photon
                $\longrightarrow$ 2 or 3 dimensions.
          \item Position, energy, momentum
                $\longrightarrow$ $\infty$ dimensions.
        \end{itemize}
\end{enumerate}

%==========================================================
\section{$L^2$ Space}
%==========================================================

\begin{definition}[$L^2$ Space]
  The space of \emph{square-integrable functions} is
  \[
    L^2(\R) = \left\{\, f : \R \to \C \;\middle|\;
               \int_{-\infty}^{\infty} |f(x)|^2\,dx < \infty \,\right\}.
  \]
  This is an infinite-dimensional Hilbert space.  Functions in $L^2$ are the
  natural setting for quantum wavefunctions.
\end{definition}

\begin{remark}
  In $L^2$, the vector $\ket{v} = c_1\ket{e_1} + c_2\ket{e_2} + \cdots$
  has basis elements $\ket{e_i}$ and coefficients $c_i$ that tell how much
  each basis element contributes to reconstruct the original ``vector''
  (function).  This is directly analogous to Fourier series coefficients.
\end{remark}

\chapter{Bra-Ket (Dirac) Notation}

Dirac's bra-ket notation is the standard language of quantum mechanics.  It
provides an elegant, basis-independent way to express vectors, dual vectors,
inner products, and operators.

%==========================================================
\section{Notation Table}
%==========================================================

\begin{center}
\renewcommand{\arraystretch}{1.4}
\begin{tabular}{@{}lll@{}}
  \toprule
  Symbol & Name & Meaning \\
  \midrule
  $z$ & Complex scalar & $z \in \C$ \\
  $z^*$ & Complex conjugate & $z^* \in \C^*$ \\
  $\ket{\psi}$ & Ket vector & $\ket{\psi} \in V(\C)$, the complex vector space \\
  $\bra{\psi}$ & Bra vector & $\bra{\psi} \in V^*(\C^*)$, the dual (conjugate) space \\
  $\braket{\phi}{\psi}$ & Inner product & A number (generally $\in\C$) \\
  $\ket{\phi}\otimes\ket{\psi}$ & Tensor product & A vector in $V\otimes V$ \\
  $A^*$ & Complex conjugate of matrix $A$ & Element-wise conjugation \\
  $A^T$ & Transpose & $(A^T)_{ij} = A_{ji}$ \\
  $A^\dagger$ & Hermitian conjugate & $A^\dagger = (A^T)^* = (\bar{A})^T$ \\
  \bottomrule
\end{tabular}
\end{center}

%==========================================================
\section{Ket Vectors — Axioms}
%==========================================================

Let $V(\C)$ be a complex vector space.  A ket $\ket{\psi}\in V(\C)$ represents
the \emph{state of a quantum particle}.  Ket vectors satisfy the following
seven axioms (for $\ket{\psi},\ket{\phi},\ket{\chi}\in V$ and $z,w\in\C$):

\begin{enumerate}[label=\textbf{K\arabic*.}]
  \item $\ket{\psi}+\ket{\phi} \in V$ \hfill(closure under $+$)
  \item $\ket{\psi}+\ket{\phi} = \ket{\phi}+\ket{\psi}$ \hfill(commutativity)
  \item $\ket{\psi}+(\ket{\phi}+\ket{\chi}) = (\ket{\psi}+\ket{\phi})+\ket{\chi}$ \hfill(associativity)
  \item $\ket{\psi}+\ket{0} = \ket{\psi}$ \hfill(additive identity)
  \item $\ket{\psi}+(-\ket{\psi}) = \ket{0}$ \hfill(additive inverse)
  \item $z\ket{\psi} = z\ket{\psi} \in V$ \hfill(scalar multiplication by $\C$)
  \item $z(\ket{A}+\ket{B}) = z\ket{A}+z\ket{B}$, \quad
        $(z+w)\ket{A} = z\ket{A}+w\ket{A}$ \hfill(linearity)
\end{enumerate}

A function $A(x)$ that is continuous and complex-valued, closed under addition
and complex-scalar multiplication, and satisfies the above axioms is a ket
vector.

\begin{example}[Column vector ket]
  \[
    \ket{\psi} = \begin{pmatrix}\alpha_1\\\alpha_2\\\vdots\\\alpha_n\end{pmatrix}, \qquad
    \ket{\psi}+\ket{\phi} = \begin{pmatrix}\alpha_1+\beta_1\\\alpha_2+\beta_2\end{pmatrix}, \qquad
    z\ket{\psi} = \begin{pmatrix}z\alpha_1\\z\alpha_2\end{pmatrix}.
  \]
\end{example}

%==========================================================
\section{Bra Vectors (Dual Space)}
%==========================================================

\begin{definition}[Bra Vector]
  For every ket $\ket{a} \in V$ there exists a unique \emph{bra}
  $\bra{a} \in V^*$ in the dual space (complex conjugate vector space):
  \[
    \ket{a} = \begin{pmatrix}\alpha_1\\\alpha_2\\\vdots\\\alpha_n\end{pmatrix}
    \quad\longleftrightarrow\quad
    \bra{a} = \bigl(\alpha_1^*\; \alpha_2^*\; \cdots\; \alpha_n^*\bigr).
  \]
\end{definition}

\begin{proposition}[Duality under scaling]
  If $z\ket{A} = \ket{B}$, then $\bra{A}z^* = \bra{B}$, i.e.,
  $z\ket{A} \Rightarrow \bra{A}z^*$.
\end{proposition}

\begin{remark}
  Bra vectors also satisfy the seven axioms above (they live in the dual
  complex-conjugate space $V^*(\C^*)$), and
  $\forall\, \ket{a}\in V\; \exists!\, \bra{a} \in V^*$.
\end{remark}

%==========================================================
\section{Inner Product in Bra-Ket Notation}
%==========================================================

\begin{definition}
  The inner product of $\ket{\phi}$ and $\ket{\psi}$ is
  \[
    \braket{\phi}{\psi} \in \C.
  \]
  Properties:
  \begin{enumerate}
    \item \textbf{Linearity:}
          $\bra{\psi}(\ket{\phi}+\ket{\chi}) = \braket{\psi}{\phi}+\braket{\psi}{\chi}$.
    \item \textbf{Conjugate property:}
          $\braket{A}{B} = \braket{B}{A}^*$.
    \item \textbf{Reality of self-inner-product:}
          $\braket{A}{A} \in \R_{\geq 0}$ (with some exceptions
          involving complex phases).
  \end{enumerate}
\end{definition}

%==========================================================
\section{Length of a Vector in Bra-Ket Notation}
%==========================================================

For $z = x+iy$,
\[
  zz^* = (x+iy)(x-iy) = x^2 + y^2,
\]
so the norm of $\ket{\psi}=(\alpha_1,\alpha_2,\alpha_3)^T$ is
\[
  \|\ket{\psi}\| = \sqrt{\alpha_1^2 + \alpha_2^2 + \alpha_3^2}.
\]
A vector of length~1 is a \textbf{unit vector}.

\begin{example}[Standard basis kets for a qubit]
  \begin{alignat*}{2}
    \ket{\uparrow} &= \begin{pmatrix}1\\0\end{pmatrix}, \quad &
    \ket{\downarrow} &= \begin{pmatrix}0\\1\end{pmatrix}, \\[6pt]
    \ket{\rightarrow} &= \frac{1}{\sqrt{2}}\begin{pmatrix}1\\-1\end{pmatrix}, \quad &
    \ket{\leftarrow} &= \frac{1}{\sqrt{2}}\begin{pmatrix}1\\1\end{pmatrix}, \\[6pt]
    \ket{\nearrow} &= \begin{pmatrix}1/2\\-\sqrt{3}/2\end{pmatrix}, \quad &
    \ket{\swarrow} &= \begin{pmatrix}\sqrt{3}/2\\1/2\end{pmatrix}.
  \end{alignat*}
  These all have unit norm. The standard basis $\{\ket{\uparrow},\ket{\downarrow}\}$
  is used for the spin of an electron.
\end{example}

%==========================================================
\section{Basis, Linear Combination, and Linear Dependence}
%==========================================================

\begin{definition}[Basis]
  A set $\{|i\rangle\}$ of vectors in $V(F)$ is a \emph{basis} if it
  \begin{enumerate}
    \item \textbf{Spans} $V$ (every vector in $V$ is a linear combination
          of basis vectors), and
    \item is \textbf{linearly independent}.
  \end{enumerate}
  An \emph{orthonormal basis} additionally satisfies $\langle i|j\rangle =
  \delta_{ij}$.
\end{definition}

\begin{theorem}[Expansion in an orthonormal basis]
  If $\{|i\rangle\}$ is an orthonormal basis of $V(F)$, then any
  $\ket{A}\in V$ can be written
  \[
    \ket{A} = \sum_i \braket{i}{A}\,\ket{i},
  \]
  where the coefficient $\braket{i}{A}$ is the \emph{component} of $\ket{A}$
  along $\ket{i}$.
\end{theorem}

\begin{proof}
  Write $\ket{A} = \sum_i \alpha_i\ket{i}$.  Apply $\bra{j}$ to both sides:
  \[
    \braket{j}{A} = \sum_i \alpha_i \braket{j}{i} = \sum_i \alpha_i\,\delta_{ji} = \alpha_j. \qed
  \]
\end{proof}

\begin{definition}[Linear Dependence and Independence]
  A set of vectors $\{\ket{v_1},\ldots,\ket{v_n}\}$ is
  \emph{linearly dependent} if there exist scalars $c_1,\ldots,c_n$, not all
  zero, such that
  \[
    c_1\ket{v_1} + c_2\ket{v_2} + \cdots + c_n\ket{v_n} = 0.
  \]
  If the only solution is $c_1=c_2=\cdots=c_n=0$, the set is
  \emph{linearly independent}.
\end{definition}

\begin{proposition}[Dimension]
  Any two linearly independent sets that span $V$ have the same cardinality,
  which defines the \emph{dimension} of $V$.  In quantum computing, Hilbert
  spaces are finite-dimensional; in quantum mechanics, infinite-dimensional.
\end{proposition}

%==========================================================
\section{Probability Amplitudes}
%==========================================================

For an infinite-dimensional expansion
\[
  \ket{\Psi} = \sum_n a_n\ket{b_n},
\]
the \emph{probability amplitude} that $\ket{\Psi}$ ``jumps'' to state
$\ket{b_k}$ upon measurement is $\braket{b_k}{\Psi}$, and the probability
is
\[
  P(\text{outcome }k) = |\braket{b_k}{\Psi}|^2 = \braket{b_k}{\Psi}\braket{\Psi}{b_k}.
\]

\chapter{Matrices}

A matrix is a rectangular array of numbers.  In quantum mechanics, linear
operators on finite-dimensional Hilbert spaces are represented by matrices.

%==========================================================
\section{Fundamental Definitions}
%==========================================================

\begin{definition}[Square Matrix]
  An $n\times n$ matrix $A$ with elements $a_{ij}$ ($1\leq i,j\leq n$).
\end{definition}

\begin{definition}[Row Matrix (Bra)]
  A $1\times n$ matrix:
  $\bra{a} = \begin{bmatrix} a & b & c \end{bmatrix}$.
\end{definition}

\begin{definition}[Column Matrix (Ket)]
  An $n\times 1$ matrix:
  $\ket{a} = \begin{bmatrix} a \\ b \\ c \end{bmatrix}$.
\end{definition}

\begin{definition}[Identity Matrix]
  The $n\times n$ identity matrix $I$ has $I_{ij}=\delta_{ij}$:
  \[
    I = \begin{bmatrix} 1 & 0 & 0 \\ 0 & 1 & 0 \\ 0 & 0 & 1 \end{bmatrix}.
  \]
  Note $AI = A = IA$ for any $n\times n$ matrix $A$.
  In general $AB \neq BA$.
\end{definition}

\begin{definition}[Transpose]
  The transpose $A^T$ of $A$ has $(A^T)_{ij} = A_{ji}$:
  \[
    A = \begin{bmatrix} a & b & c \\ d & e & f \\ g & h & i \end{bmatrix}
    \implies
    A^T = \begin{bmatrix} a & d & g \\ b & e & h \\ c & f & i \end{bmatrix}.
  \]
\end{definition}

\begin{definition}[Trace]
  The \emph{trace} of a square matrix $A$ is the sum of its diagonal elements:
  $\Tr(A) = \sum_i A_{ii}$.
\end{definition}

%==========================================================
\section{Special Matrices}
%==========================================================

\subsection{Symmetric Matrix}

\begin{definition}
  $A$ is \emph{symmetric} if $A^T = A$.
\end{definition}

\begin{example}
  $A = \begin{pmatrix}1&0\\0&1\end{pmatrix}$,
  $A^T = \begin{pmatrix}1&0\\0&1\end{pmatrix} = A$. \checkmark
\end{example}

\subsection{Hermitian Matrix}

\begin{definition}
  $A$ is \emph{Hermitian} (self-adjoint) if $A^\dagger = A$, where
  $A^\dagger = (\bar{A})^T$ is the Hermitian conjugate.
\end{definition}

\begin{remark}
  The \emph{diagonal elements of a Hermitian matrix are always real}, since
  $(A^\dagger)_{ii} = \bar{A}_{ii} = A_{ii} \Rightarrow A_{ii} \in \R$.
\end{remark}

\begin{example}
  \[
    A = \begin{pmatrix}3 & 1-i \\ 1+i & -2\end{pmatrix}, \quad
    \bar{A} = \begin{pmatrix}3 & 1+i \\ 1-i & -2\end{pmatrix}, \quad
    (\bar{A})^T = \begin{pmatrix}3 & 1-i \\ 1+i & -2\end{pmatrix} = A. \quad\checkmark
  \]
\end{example}

\subsection{Orthogonal Matrix}

\begin{definition}
  $A$ is \emph{orthogonal} if $AA^T = I = A^TA$.
  The rows (and columns) of $A$ form an orthonormal set.
\end{definition}

\begin{example}[Rotation matrix]
  \[
    A = \begin{pmatrix}\cos\alpha & \sin\alpha \\ -\sin\alpha & \cos\alpha\end{pmatrix},
    \quad
    A^T = \begin{pmatrix}\cos\alpha & -\sin\alpha \\ \sin\alpha & \cos\alpha\end{pmatrix},
  \]
  \[
    AA^T = \begin{pmatrix}\cos^2\!\alpha+\sin^2\!\alpha & 0 \\ 0 & \sin^2\!\alpha+\cos^2\!\alpha\end{pmatrix}
         = \begin{pmatrix}1&0\\0&1\end{pmatrix} = I. \quad\checkmark
  \]
\end{example}

\begin{example}[$3\times 3$ orthogonal matrix]
  \[
    A = \begin{pmatrix}
      1/3 & 2/3 & -2/3 \\
      -2/3 & 2/3 & 1/3 \\
      2/3 & 1/3 & 2/3
    \end{pmatrix}, \qquad AA^T = I. \quad\checkmark
  \]
\end{example}

\subsection{Unitary Matrix}

\begin{definition}
  $A$ is \emph{unitary} if $AA^\dagger = I = A^\dagger A$.
  Unitary matrices are the complex generalisation of orthogonal matrices;
  they preserve inner products.
\end{definition}

\begin{example}
  \[
    A = \frac{1}{\sqrt{2}}\begin{pmatrix}1 & 1 \\ i & -i\end{pmatrix},
    \quad
    A^\dagger = \frac{1}{\sqrt{2}}\begin{pmatrix}1 & -i \\ 1 & i\end{pmatrix},
    \quad
    AA^\dagger = I. \quad\checkmark
  \]
\end{example}

%==========================================================
\section{Checking if a Set Forms an Orthonormal Basis}
%==========================================================

\begin{proposition}
  A set $\mathcal{A} = \{|b_i\rangle\}$ forms an orthonormal basis if and
  only if the matrix $\left[\langle b_i|b_j\rangle\right]_{i,j} = I$.
\end{proposition}

\begin{proof}
  In matrix form:
  \[
    \underbrace{\begin{bmatrix}\bra{b_1}\\\bra{b_2}\\\vdots\\\bra{b_n}\end{bmatrix}}_{n\times 1}
    \underbrace{\begin{bmatrix}\ket{b_1} & \ket{b_2} & \cdots & \ket{b_n}\end{bmatrix}}_{1\times n}
    = \begin{bmatrix}
        \braket{b_1}{b_1} & \braket{b_1}{b_2} & \cdots \\
        \braket{b_2}{b_1} & \braket{b_2}{b_2} & \cdots \\
        \vdots & & \ddots
      \end{bmatrix} = I. \qed
  \]
  This is equivalent to checking: (i) normalization $\braket{b_i}{b_i}=1$,
  and (ii) orthogonality $A^\dagger A = I$.
\end{proof}

\chapter{Linear Operators and Eigenvalues}

%==========================================================
\section{Linear Operators}
%==========================================================

In quantum mechanics:
\begin{itemize}
  \item \textbf{States} of a system are represented by vectors in a
        vector space (Hilbert space).
  \item \textbf{Physical observables} (energy, position, momentum, angular
        momentum) are represented by \emph{linear operators} on that space.
\end{itemize}

\begin{definition}[Linear Operator]
  A map $M: V(F) \to V(F)$ is a \emph{linear operator} if it satisfies
  (for all $\ket{A},\ket{B}\in V$ and $z\in F$):
  \begin{enumerate}[label=\textbf{LO\arabic*.}]
    \item $M\bigl(z\ket{A}\bigr) = z\,M\ket{A}$,
    \item $M\bigl(\ket{A}+\ket{B}\bigr) = M\ket{A}+M\ket{B}$.
  \end{enumerate}
  Think of $M$ as a machine: input goes in, output comes out.  If no output
  is produced, $M$ is not a linear operator.
\end{definition}

\begin{remark}
  Not every operator is linear.  Linear operators are \emph{associated with}
  a vector space but are not themselves vectors.
\end{remark}

%==========================================================
\section{Matrix Representation of Linear Operators}
%==========================================================

\begin{proposition}
  Given an orthonormal basis $\{|i\rangle\}$ of $V$, a linear operator $M$
  is represented by the matrix with entries
  \[
    m_{ki} = \langle k \mid M \mid i \rangle.
  \]
  The action $M\ket{A} = \ket{B}$ becomes
  \[
    \begin{pmatrix} m_{11} & m_{12} & m_{13} \\ m_{21} & m_{22} & m_{23} \\
    m_{31} & m_{32} & m_{33} \end{pmatrix}
    \begin{pmatrix} a_1 \\ a_2 \\ a_3 \end{pmatrix}
    = \begin{pmatrix} b_1 \\ b_2 \\ b_3 \end{pmatrix},
    \qquad
    \sum_i a_i\, m_{ki} = b_k.
  \]
\end{proposition}

\begin{remark}
  The dimensions of the matrix $M$ depend on the choice of basis.  The
  relationship between vectors and operators is \emph{independent} of basis
  in abstract notation but acquires a concrete matrix form once a basis is
  fixed.
\end{remark}

\paragraph{Special operators.}
\begin{itemize}
  \item \textbf{Identity operator:} $I_V\ket{v} = \ket{v}$,\quad $\ket{v}\in V(F)$.
  \item \textbf{Zero operator:} $O\ket{v} = 0$,\quad $\ket{v}\in V(F)$.
  \item \textbf{Composition:} If $V: A\to B$ and $W: B\to C$ are linear,
        then $WV\ket{a} = W(V\ket{a})$ is linear, and $AB(\ket{v}) = A(B\ket{v})$.
\end{itemize}

%==========================================================
\section{Eigenvalues and Eigenvectors}
%==========================================================

\begin{definition}[Eigenvalue / Eigenvector]
  A non-zero vector $\ket{\lambda}$ is an \emph{eigenvector} of $M$ with
  \emph{eigenvalue} $\lambda \in \C$ if
  \[
    \boxed{M\ket{\lambda} = \lambda\ket{\lambda}.}
  \]
  The operator $M$ preserves the \emph{direction} of $\ket{\lambda}$ (only
  scaling it by $\lambda$).
\end{definition}

\begin{remark}
  In general, a linear operator \emph{changes} the direction of its input
  vector.  Eigenvectors are special inputs for which only scaling occurs.
\end{remark}

\begin{example}
  Let $M = \begin{pmatrix}1&2\\2&1\end{pmatrix}$.
  \begin{enumerate}
    \item $\ket{v} = \begin{pmatrix}1\\1\end{pmatrix}$:
          $M\ket{v} = \begin{pmatrix}3\\3\end{pmatrix} = 3\ket{v}$.
          Eigenvalue $\lambda = 3$. \checkmark
    \item $\ket{u} = \begin{pmatrix}1\\-1\end{pmatrix}$:
          $M\ket{u} = \begin{pmatrix}-1\\1\end{pmatrix} = -1\ket{u}$.
          Eigenvalue $\lambda = -1$. \checkmark
    \item Let $N = \begin{pmatrix}0&-1\\1&0\end{pmatrix}$,
          $\ket{u'} = \begin{pmatrix}1\\i\end{pmatrix}$:
          $N\ket{u'} = \begin{pmatrix}-i\\1\end{pmatrix} = -i\ket{u'}$.
          Eigenvalue $\lambda = -i$. \checkmark
  \end{enumerate}
\end{example}

%==========================================================
\section{Hermitian Conjugate and Its Role in Quantum Mechanics}
%==========================================================

\begin{proposition}
  For a linear operator $M$, the map in dual space is given by:
  \[
    M\ket{A} = \ket{B} \quad\Longleftrightarrow\quad \bra{A}M^\dagger = \bra{B},
  \]
  so $M^\dagger$ is the Hermitian conjugate (adjoint) of $M$.  In matrix form,
  \[
    M = \begin{pmatrix}a&b\\c&d\end{pmatrix}
    \implies
    M^\dagger = \begin{pmatrix}a^*&c^*\\b^*&d^*\end{pmatrix}.
  \]
\end{proposition}

\begin{proof}[Derivation]
  Starting from $M\ket{A}=\ket{B}$, expand in orthonormal basis:
  $M\sum_i a_i\ket{i} = \sum_i b_i\ket{i}$.
  Apply $\bra{j}$: $\sum_i a_i \langle j|M|i\rangle = b_j$, so $m_{ji}$ are
  the matrix elements.  In dual space with complex-conjugate coefficients,
  $m_{ji}^*$ appear, giving $(M^\dagger)_{ij} = m_{ji}^* = (M^T)^*_{ij}$,
  i.e.\ $M^\dagger = (\bar M)^T$. \qed
\end{proof}

%==========================================================
\section{Observables and Hermitian Operators}
%==========================================================

\begin{theorem}[Observables are Hermitian]\label{thm:hermitian-real}
  Observables in quantum mechanics are represented by Hermitian operators
  ($M = M^\dagger$), and \emph{all eigenvalues of a Hermitian operator are
  real}.
\end{theorem}

\begin{proof}
  Let $M\ket{\lambda} = \lambda\ket{\lambda}$.  Since $M = M^\dagger$,
  \[
    \bra{\lambda}M^\dagger = \lambda^*\bra{\lambda}.
  \]
  Multiply $\bra{\lambda}$ on the left of $M\ket{\lambda}=\lambda\ket{\lambda}$:
  \[
    \braket{\lambda}{M|\lambda} = \lambda\braket{\lambda}{\lambda}.
  \]
  Multiply $M\ket{\lambda}=\lambda\ket{\lambda}$ on the right with $\bra{\lambda}$
  and use $M=M^\dagger$:
  \[
    \braket{\lambda}{M|\lambda} = \lambda^*\braket{\lambda}{\lambda}.
  \]
  Subtracting: $0 = (\lambda^*-\lambda)\braket{\lambda}{\lambda}$.
  Since $\ket{\lambda}\neq 0$ we have $\braket{\lambda}{\lambda}>0$, hence
  $\lambda^* = \lambda$, i.e.\ $\lambda \in \R$. \qed
\end{proof}

\begin{remark}
  This is physically essential: the outcome of any experiment is always a real
  number, and observables correspond to Hermitian operators precisely to
  guarantee this.
\end{remark}

%==========================================================
\section{Cauchy--Schwarz Inequality}
%==========================================================

\begin{theorem}[Cauchy--Schwarz Inequality]
  For any two vectors $\ket{u},\ket{v}$ in a Hilbert space $\mathcal{H}$,
  \[
    \abs{\braket{u}{v}}^2 \leq \braket{u}{u}\,\braket{v}{v}.
  \]
\end{theorem}

\begin{remark}
  The Cauchy--Schwarz inequality is fundamental to Hilbert space theory.  It
  bounds how much any two vectors can ``overlap'' and is a key ingredient in
  proving the triangle inequality for the induced norm.
\end{remark}

\chapter{Fourier Analysis}

Fourier analysis is the mathematical heart of quantum mechanics.  The Fourier
transform is the mathematical heart of Heisenberg's uncertainty principle
(Rajan Chopra).  It decomposes functions into their constituent frequencies,
exactly as an orthonormal basis decomposes vectors.

%==========================================================
\section{Motivation: Seeing Functions Differently}
%==========================================================

The Fourier transform is a way of viewing a function from a different
perspective — the \emph{frequency domain} instead of the \emph{time domain}.
Just as $\hat{\imath},\hat{\jmath},\hat{k}$ span all of 3D space, the
exponential functions $e^{i n\omega_0 t}$ span $L^2$ space.

%==========================================================
\section{Fourier Series}
%==========================================================

\begin{definition}[Periodic Function]
  A function $f: \R \to \C$ is \emph{periodic} with period $T$ if
  $f(t) = f(t + T)$ for all $t$.
\end{definition}

\begin{theorem}[Fourier Series]
  Any periodic function $f(t)$ with period $T$ can be expressed as a sum of
  sines and cosines (each with its own amplitude and frequency):
  \[
    f(t) = a_0 + \sum_{n=1}^{\infty}
           \bigl[\, a_n \sin(n\omega_0 t) + b_n \cos(n\omega_0 t)\,\bigr],
  \]
  where $\omega_0 = 2\pi / T$ is the \emph{fundamental angular frequency} and
  $\omega = 2\pi\nu$ ($\nu$ = frequency in waves per second).
  Sines and cosines act as the \emph{basis functions}.
\end{theorem}

\begin{example}
  $f(t) = \sin\omega t + \tfrac{1}{2}\sin 2\omega t + \tfrac{3}{2}\cos\omega t$.
  Each term is one frequency component, so this function has been expressed in
  its ``Fourier basis.''
\end{example}

\subsection{Why Use Sines and Cosines?}

\begin{enumerate}
  \item \textbf{Orthogonality:} Sine and cosine are mutually independent
        (like orthogonal vectors in a vector space).
  \item \textbf{Completeness:} The set $\{\sin(n\omega_0 t), \cos(n\omega_0 t)\}$
        is a complete set — it can generate a wide variety of functions,
        especially from $L^2$.
\end{enumerate}

%==========================================================
\section{Complex Exponential Form}
%==========================================================

Using Euler's formula $e^{i\theta} = \cos\theta + i\sin\theta$:
\[
  \cos\theta = \frac{e^{i\theta}+e^{-i\theta}}{2}, \qquad
  \sin\theta = \frac{e^{i\theta}-e^{-i\theta}}{2i}.
\]

Substituting into the Fourier series and collecting terms, one obtains the
\textbf{complex exponential Fourier series}:

\begin{equation}\label{eq:complex-fourier}
  \boxed{f(t) = \sum_{n=-\infty}^{\infty} C_n\, e^{in\omega_0 t},}
\end{equation}

where $C_n \in \C$ are the \emph{complex Fourier coefficients}.

\paragraph{Why $e^{i\omega t}$?}  Three reasons:
\begin{enumerate}
  \item \textbf{Orthogonality:} $e^{i\omega t},\, e^{2i\omega t},\ldots$ are
        all mutually independent.
  \item \textbf{Completeness:} Any well-behaved function is a linear
        combination of $\{e^{in\omega_0 t}\}$.
  \item \textbf{Periodic nature:} $e^{i\omega t}$ traces a circle in the
        complex plane as $t$ varies.
\end{enumerate}

%==========================================================
\section{Derivation of Fourier Coefficients}
%==========================================================

\begin{theorem}[Fourier Coefficients]
  The $n$-th complex Fourier coefficient of a periodic function $f(t)$ with
  period $T$ is
  \[
    \boxed{C_n = \frac{1}{T}\int_0^T f(t)\, e^{-in\omega_0 t}\,dt.}
  \]
\end{theorem}

\begin{proof}
  Start from~\eqref{eq:complex-fourier}: $f(t) = \sum_{k=-\infty}^{\infty} C_k e^{ik\omega_0 t}$.

  \textbf{Step 1.} Multiply both sides by $e^{-in\omega_0 t}$ (for a fixed
  integer $n$) and integrate over one period $[0,T]$:
  \[
    \int_0^T e^{-in\omega_0 t} f(t)\,dt
    = \int_0^T e^{-in\omega_0 t} \sum_{k=-\infty}^{\infty} C_k e^{ik\omega_0 t}\,dt.
  \]

  \textbf{Step 2.} Exchange sum and integral (Fubini's theorem):
  \[
    = \sum_{k=-\infty}^{\infty} C_k \int_0^T e^{i(k-n)\omega_0 t}\,dt.
  \]

  \textbf{Step 3.} Use orthogonality of complex exponentials: any two
  exponentials $e^{ik\omega_0 t}$ and $e^{in\omega_0 t}$ are orthogonal over
  one period (their inner product is zero if $k\neq n$).  Explicitly, if
  $k \neq n$:
  \begin{align*}
    \int_0^T e^{i(k-n)\omega_0 t}\,dt
    &= \left[\frac{e^{i(k-n)\omega_0 t}}{i(k-n)\omega_0}\right]_0^T
     = \frac{e^{i(k-n)\cdot 2\pi}-1}{i(k-n)\omega_0} \\
    &= \frac{\cos 2\pi(k-n)+i\sin 2\pi(k-n)-1}{i(k-n)\omega_0}
     = \frac{1+0-1}{i(k-n)\omega_0} = 0.
  \end{align*}

  \textbf{Step 4.} Only the $k=n$ term survives:
  \[
    \int_0^T e^{-in\omega_0 t} f(t)\,dt = C_n \int_0^T 1\,dt = C_n T.
  \]

  \textbf{Step 5.} Solve for $C_n$:
  \[
    C_n = \frac{1}{T}\int_0^T f(t)\,e^{-in\omega_0 t}\,dt. \qed
  \]
\end{proof}

%==========================================================
\section{Fourier Transform}
%==========================================================

\begin{definition}[Fourier Transform]
  For a \emph{non-periodic} function $f(t)$ (thought of as having a
  continuous spectrum of frequencies), the \textbf{Fourier transform} is
  \[
    \boxed{\tilde{f}(\omega) = \int_{-\infty}^{\infty} f(t)\, e^{-i\omega t}\,dt.}
  \]
  This transforms from the \emph{time domain} to the \emph{frequency domain}.
  $\tilde{f}(\omega)$ is the \emph{amplitude} of the frequency component
  $\omega$ in the original function — it tells us how much $e^{i\omega t}$
  contributes to $f(t)$.
\end{definition}

\begin{definition}[Inverse Fourier Transform]
  The original function is recovered by
  \[
    \boxed{f(t) = \frac{1}{2\pi}\int_{-\infty}^{\infty} \tilde{f}(\omega)\, e^{i\omega t}\,d\omega.}
  \]
  This transforms from the \emph{frequency domain} back to the \emph{time
  domain}.  The factor $\frac{1}{2\pi}$ is the normalizing factor.
\end{definition}

\begin{remark}
  A non-periodic function can be thought of as having a \emph{continuous}
  spectrum of frequencies.  It can be represented as a continuous sum
  (integral) of sines, cosines, or exponentials $e^{i\omega t}$,
  over (possibly) all frequencies.
\end{remark}

\begin{example}[Gaussian function]
  Let $f(t) = e^{-t^2}$.  Then
  \[
    \tilde{f}(\omega) = \int_{-\infty}^{\infty} e^{-t^2} e^{-i\omega t}\,dt
                      = \frac{1}{\sqrt{\pi}}\,e^{-\omega^2/4},
  \]
  so
  \[
    e^{-t^2} = \int_{-\infty}^{\infty}
               \underbrace{\frac{1}{\sqrt{\pi}} e^{-\omega^2/4}}_{\text{coeff.}}
               \underbrace{e^{i\omega t}}_{\text{basis}}\,d\omega.
  \]
  A Gaussian in time is a Gaussian in frequency — a hallmark of the
  uncertainty principle.
\end{example}

%==========================================================
\section{Connection to Quantum Mechanics}
%==========================================================

The Fourier transform is the heart of \textbf{Heisenberg's Uncertainty
Principle}.

\begin{itemize}
  \item A sharply localized (narrow) wavepacket in position-space
        corresponds to a \emph{spread-out} wavefunction in momentum-space,
        and vice versa.
  \item Mathematically: $\Delta x\, \Delta p \geq \hbar/2$.
  \item The Fourier transform relates the two representations.
\end{itemize}

The analogy with linear algebra is exact:
\[
  \underbrace{\ket{v} = \sum_i c_i\ket{e_i}}_{\text{vector expansion}}
  \quad\longleftrightarrow\quad
  \underbrace{f(t) = \int \tilde{f}(\omega)\,e^{i\omega t}\,d\omega}_{\text{Fourier expansion}},
\]
with $e^{i\omega t}$ playing the role of the orthonormal basis functions and
$\tilde{f}(\omega)$ playing the role of the expansion coefficients $c_i$.

\chapter{Outer Product, Completeness, and Projection Operators}

%==========================================================
\section{Outer Product}
%==========================================================

\begin{definition}[Outer Product]
  The \emph{outer product} of $\ket{\psi}\in\R^n$ and $\ket{\phi}\in\R^m$
  is the $n\times m$ matrix
  \[
    \ket{\psi}\bra{\phi} \in \R^{n+m},
  \]
  formed by
  \[
    \underbrace{(\;\cdot\;)}_{n\times 1}
    \underbrace{(\;\cdot\;)}_{1\times m}
    = \underbrace{(\;\cdot\;)}_{n\times m}.
  \]
\end{definition}

\begin{remark}
  Contrast with the \emph{inner product} $\braket{\phi}{\psi}$, which
  collapses into a single number (scalar).  The outer product
  $\ket{\psi}\bra{\phi}$ preserves the full structure of both vectors and
  produces a matrix (higher-dimensional object).
\end{remark}

\paragraph{Physical interpretation.}
\begin{itemize}
  \item Outer product $\Rightarrow$ \textbf{Projection operator} in quantum
        mechanics.
  \item $\ket{\psi}\bra{\phi}$ is a way of ``spreading out'' the interaction
        between two vectors in a higher-dimensional space.
  \item Unlike the dot product (which collapses into a scalar), the outer
        product retains both vectors' properties.
\end{itemize}

\begin{example}
  $(\ket{\psi}\bra{\phi})\ket{\chi} = \ket{\psi}\braket{\phi}{\chi}$:
  first $\bra{\phi}$ is projected onto $\ket{\chi}$ yielding a number
  $\braket{\phi}{\chi}$, then $\ket{\psi}$ is scaled by that number.
  The projection operator thus performs \emph{projection then scaling}.
\end{example}

%==========================================================
\section{Completeness Relation}
%==========================================================

\begin{theorem}[Completeness Relation]
  Let $\{|i\rangle\}$ be an orthonormal basis of $V(F)$.  Then
  \[
    \boxed{\sum_i \ket{i}\bra{i} = I,}
  \]
  the identity operator on $V$.
\end{theorem}

\begin{proof}
  Let $\ket{v}\in V(F)$ be arbitrary.  Since $\{|i\rangle\}$ is orthonormal,
  $\ket{v} = \sum_i v_i\ket{i}$ with $v_i = \braket{i}{v}$.  Therefore:
  \[
    \left(\sum_i \ket{i}\bra{i}\right)\ket{v}
    = \sum_i \ket{i}\braket{i}{v}
    = \sum_i v_i\ket{i}
    = \ket{v}.
  \]
  Since this holds for all $\ket{v}$, $\sum_i\ket{i}\bra{i} = I$. \qed
\end{proof}

\begin{remark}
  The completeness relation is the outer-product analogue of the statement
  that any vector can be written as a linear combination of basis vectors.
  It is used constantly to insert resolutions of the identity in quantum
  mechanical calculations.
\end{remark}

%==========================================================
\section{Representing Linear Operators as Outer Products}
%==========================================================

\begin{theorem}[Outer Product Representation of a Linear Operator]
  Let $A: V \to W$ be a linear operator, with $\{|v_i\rangle\}$ an
  orthonormal basis of $V$ and $\{|w_j\rangle\}$ an orthonormal basis of $W$.
  Then
  \[
    A = \sum_{i,j} \ket{v_i}\bra{w_j}\, \langle v_i|A|w_j\rangle.
  \]
  Equivalently, by inserting two completeness relations $I_V$ and $I_W$:
  \[
    A = I_V\, A\, I_W
      = \left(\sum_i \ket{v_i}\bra{v_i}\right) A
        \left(\sum_j \ket{w_j}\bra{w_j}\right)
      = \sum_{i,j} \ket{v_i}\underbrace{\langle v_i|A|w_j\rangle}_{\text{matrix element}}\bra{w_j}.
  \]
\end{theorem}

\begin{remark}
  Linear operators can be represented in three equivalent ways:
  \begin{enumerate}
    \item \textbf{Abstract notation:} $A$
    \item \textbf{Matrix notation:} $[A_{ij}]$ with $A_{ij}=\bra{i}A\ket{j}$
    \item \textbf{Outer product:} $A = \sum_{ij} \ket{v_i}\langle v_i|A|w_j\rangle\bra{w_j}$
  \end{enumerate}
  The choice depends on the calculation.
\end{remark}

%==========================================================
\section{Cauchy--Schwarz Inequality (Inner Product Form)}
%==========================================================

\begin{theorem}[Cauchy--Schwarz Inequality]
  For any $\ket{u},\ket{v}$ in a Hilbert space $\mathcal{H}$:
  \[
    |\braket{u}{v}|^2 \leq \braket{u}{u}\cdot\braket{v}{v}.
  \]
  This bounds the squared inner product by the product of norms-squared.
\end{theorem}

\begin{remark}
  The Cauchy--Schwarz inequality:
  \begin{itemize}
    \item Ensures the inner product is well-defined on $\mathcal{H}$.
    \item Is equivalent to the statement $|\cos\theta|\leq 1$ in geometry.
    \item Is a cornerstone of functional analysis and quantum mechanics.
  \end{itemize}
\end{remark}


\end{document}
