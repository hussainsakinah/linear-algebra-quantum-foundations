\chapter{Vector Spaces}

%==========================================================
\section{What is ``Space'' in Mathematics?}
%==========================================================

In everyday language, \emph{space} is physical.  In mathematics, a
\emph{space} is a non-empty set equipped with a \textbf{mathematical
structure} — a collection of axioms, theorems, and definitions that constrain
the set's behaviour.

\begin{example}
  $A = \{1, 2, 3\}$ is an \emph{ordinary} (simple) set with no additional
  structure.  By contrast, $(\R, +)$ follows the axioms of an Abelian group
  and is therefore a \emph{space}.
\end{example}

%==========================================================
\section{Definition of a Vector Space}
%==========================================================

\begin{definition}[Vector Space]\label{def:vector-space}
  A \emph{vector space} over a field $F$ is a non-empty set $V$ together with
  two operations:
  \begin{itemize}
    \item \textbf{Vector addition:} $+: V\times V \to V$,
    \item \textbf{Scalar multiplication:} $\cdot: F\times V \to V$,
  \end{itemize}
  satisfying the following nine axioms for all $\mathbf{u},\mathbf{v},\mathbf{w}\in V$
  and $a,b\in F$:

  \begin{enumerate}[label=\textbf{VS\arabic*.}]
    \item $\mathbf{u}+\mathbf{v} = \mathbf{v}+\mathbf{u}$
          \hfill(commutativity of $+$)
    \item $(\mathbf{u}+\mathbf{v})+\mathbf{w} = \mathbf{u}+(\mathbf{v}+\mathbf{w})$
          \hfill(associativity of $+$)
    \item $\exists\, \mathbf{0}\in V$ such that $\mathbf{0}+\mathbf{u}=\mathbf{u}+\mathbf{0}=\mathbf{u}$
          \hfill(additive identity)
    \item $\forall\,\mathbf{u}\in V\;\exists\,(-\mathbf{u})\in V$ such that
          $\mathbf{u}+(-\mathbf{u})=\mathbf{0}$
          \hfill(additive inverse)
    \item $1\cdot\mathbf{u} = \mathbf{u}$ \hfill(scalar identity)
    \item $a\cdot(b\cdot\mathbf{u}) = (ab)\cdot\mathbf{u}$
          \hfill(compatibility of scalar multiplication)
    \item $(a+b)\mathbf{u} = a\mathbf{u} + b\mathbf{u}$
          \hfill(distributivity over scalar addition)
    \item $a(\mathbf{u}+\mathbf{v}) = a\mathbf{u} + a\mathbf{v}$
          \hfill(distributivity over vector addition)
    \item $\exists\,\mathbf{0}\in V$ such that $\mathbf{u}\cdot\mathbf{0}=\mathbf{0}$
          \hfill(zero vector)
  \end{enumerate}
\end{definition}

\begin{remark}
  Axioms VS1--VS4 make $(V,+)$ an Abelian group.  The scalar multiplication
  axioms VS5--VS8 couple the field structure to the group structure.
\end{remark}

%==========================================================
\section{Examples of Vector Spaces}
%==========================================================

\begin{example}
  The following are all vector spaces over appropriate fields:
  \begin{enumerate}
    \item \textbf{Polynomials of degree $\leq 1$:}
          $u = x+1$, $v = 2x-3$; then $u+v = 3x-2$ and $5u = 5x+5$,
          both first-degree polynomials. \checkmark
    \item \textbf{$2\times 2$ matrices:}
          $u = \begin{psmallmatrix}2&1\\1&3\end{psmallmatrix}$,
          $v = \begin{psmallmatrix}-1&0\\1&1\end{psmallmatrix}$;
          addition and scalar multiplication are component-wise. \checkmark
    \item \textbf{Indefinite integrals:}
          $u = \int f(x)\,dx$, $v = \int g(x)\,dx$;
          $u+v = \int(f+g)(x)\,dx$. \checkmark
    \item \textbf{Derivatives:}
          $u = \tfrac{d}{dx}f(x)$, $v = \tfrac{d}{dx}g(x)$. \checkmark
    \item \textbf{Real scalars:} $u = 1$, $v = 2$. \checkmark
  \end{enumerate}
\end{example}

%==========================================================
\section{Subspaces}
%==========================================================

\begin{definition}[Subspace]
  Let $V(F)$ be a vector space.  A non-empty subset $W \subseteq V$ is a
  \emph{subspace} of $V$ if $W$ is itself a vector space over $F$ under the
  same operations.  Equivalently, $W$ is a subspace if and only if:
  \begin{enumerate}
    \item $\mathbf{0} \in W$,
    \item $\mathbf{u},\mathbf{v}\in W \Rightarrow \mathbf{u}+\mathbf{v}\in W$,
    \item $a\in F,\;\mathbf{u}\in W \Rightarrow a\mathbf{u}\in W$.
  \end{enumerate}
\end{definition}

%==========================================================
\section{Metric Spaces}
%==========================================================

\begin{definition}[Metric Space]
  Let $X$ be a non-empty set.  A function $d: X\times X \to \R$ is a
  \emph{metric} (distance function) on $X$ if, for all $x,y,z\in X$:
  \begin{enumerate}[label=(\roman*)]
    \item $d(x,y) \geq 0$ and $d(x,y)=0 \iff x=y$,
    \item $d(x,y) = d(y,x)$ \hfill(symmetry),
    \item $d(x,z) \leq d(x,y) + d(y,z)$ \hfill(triangle inequality).
  \end{enumerate}
  The pair $(X,d)$ is then called a \emph{metric space}.
\end{definition}

\begin{example}
  On $X = \R$, the function $d(x,y) = |x-y|$ is a metric (the standard
  Euclidean distance).  Any $x$ and $y$ satisfying the three conditions above
  define a metric space — they need not be numbers.
\end{example}

%==========================================================
\section{Key Group-Theoretic Theorems}
%==========================================================

\begin{theorem}[Uniqueness of Identity]
  If $(G,*)$ is a group, then the identity element $e$ is unique.
\end{theorem}

\begin{proof}
  Suppose $e$ and $e'$ are both identities.  Then
  $e = e * e' = e'$, since $e'$ is an identity for $e$ and $e$ is an
  identity for $e'$. \qed
\end{proof}

\begin{theorem}[Uniqueness of Inverses]
  In a group $(G,*)$, every element has a unique inverse.
\end{theorem}

\begin{proof}
  Let $b$ and $c$ both be inverses of $a$.  Then
  $b = b*e = b*(a*c) = (b*a)*c = e*c = c$. \qed
\end{proof}

%==========================================================
\section{Internal and External Compositions}
%==========================================================

\begin{definition}[Internal Composition]
  A composition $*$ on a non-empty set $V$ is \emph{internal} if
  \[
    \forall\, \alpha,\beta \in V \implies \alpha * \beta \in V.
  \]
  The result is uniquely determined.
\end{definition}

\begin{definition}[External Composition]
  Let $F$ be a field (of scalars) and $V$ a non-empty set.  A composition
  $\circ: F \times V \to V$ is \emph{external} if
  \[
    \forall\, a \in F,\; \alpha \in V \implies a \circ \alpha \in V,
  \]
  and the result is unique.
\end{definition}

\begin{remark}
  Vector addition is an \emph{internal} composition on $V$; scalar
  multiplication is an \emph{external} composition from $F$ into $V$.  Both
  are required to define a vector space.
\end{remark}
