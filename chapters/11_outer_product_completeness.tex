\chapter{Outer Product, Completeness, and Projection Operators}

%==========================================================
\section{Outer Product}
%==========================================================

\begin{definition}[Outer Product]
  The \emph{outer product} of $\ket{\psi}\in\R^n$ and $\ket{\phi}\in\R^m$
  is the $n\times m$ matrix
  \[
    \ket{\psi}\bra{\phi} \in \R^{n+m},
  \]
  formed by
  \[
    \underbrace{(\;\cdot\;)}_{n\times 1}
    \underbrace{(\;\cdot\;)}_{1\times m}
    = \underbrace{(\;\cdot\;)}_{n\times m}.
  \]
\end{definition}

\begin{remark}
  Contrast with the \emph{inner product} $\braket{\phi}{\psi}$, which
  collapses into a single number (scalar).  The outer product
  $\ket{\psi}\bra{\phi}$ preserves the full structure of both vectors and
  produces a matrix (higher-dimensional object).
\end{remark}

\paragraph{Physical interpretation.}
\begin{itemize}
  \item Outer product $\Rightarrow$ \textbf{Projection operator} in quantum
        mechanics.
  \item $\ket{\psi}\bra{\phi}$ is a way of ``spreading out'' the interaction
        between two vectors in a higher-dimensional space.
  \item Unlike the dot product (which collapses into a scalar), the outer
        product retains both vectors' properties.
\end{itemize}

\begin{example}
  $(\ket{\psi}\bra{\phi})\ket{\chi} = \ket{\psi}\braket{\phi}{\chi}$:
  first $\bra{\phi}$ is projected onto $\ket{\chi}$ yielding a number
  $\braket{\phi}{\chi}$, then $\ket{\psi}$ is scaled by that number.
  The projection operator thus performs \emph{projection then scaling}.
\end{example}

%==========================================================
\section{Completeness Relation}
%==========================================================

\begin{theorem}[Completeness Relation]
  Let $\{|i\rangle\}$ be an orthonormal basis of $V(F)$.  Then
  \[
    \boxed{\sum_i \ket{i}\bra{i} = I,}
  \]
  the identity operator on $V$.
\end{theorem}

\begin{proof}
  Let $\ket{v}\in V(F)$ be arbitrary.  Since $\{|i\rangle\}$ is orthonormal,
  $\ket{v} = \sum_i v_i\ket{i}$ with $v_i = \braket{i}{v}$.  Therefore:
  \[
    \left(\sum_i \ket{i}\bra{i}\right)\ket{v}
    = \sum_i \ket{i}\braket{i}{v}
    = \sum_i v_i\ket{i}
    = \ket{v}.
  \]
  Since this holds for all $\ket{v}$, $\sum_i\ket{i}\bra{i} = I$. \qed
\end{proof}

\begin{remark}
  The completeness relation is the outer-product analogue of the statement
  that any vector can be written as a linear combination of basis vectors.
  It is used constantly to insert resolutions of the identity in quantum
  mechanical calculations.
\end{remark}

%==========================================================
\section{Representing Linear Operators as Outer Products}
%==========================================================

\begin{theorem}[Outer Product Representation of a Linear Operator]
  Let $A: V \to W$ be a linear operator, with $\{|v_i\rangle\}$ an
  orthonormal basis of $V$ and $\{|w_j\rangle\}$ an orthonormal basis of $W$.
  Then
  \[
    A = \sum_{i,j} \ket{v_i}\bra{w_j}\, \langle v_i|A|w_j\rangle.
  \]
  Equivalently, by inserting two completeness relations $I_V$ and $I_W$:
  \[
    A = I_V\, A\, I_W
      = \left(\sum_i \ket{v_i}\bra{v_i}\right) A
        \left(\sum_j \ket{w_j}\bra{w_j}\right)
      = \sum_{i,j} \ket{v_i}\underbrace{\langle v_i|A|w_j\rangle}_{\text{matrix element}}\bra{w_j}.
  \]
\end{theorem}

\begin{remark}
  Linear operators can be represented in three equivalent ways:
  \begin{enumerate}
    \item \textbf{Abstract notation:} $A$
    \item \textbf{Matrix notation:} $[A_{ij}]$ with $A_{ij}=\bra{i}A\ket{j}$
    \item \textbf{Outer product:} $A = \sum_{ij} \ket{v_i}\langle v_i|A|w_j\rangle\bra{w_j}$
  \end{enumerate}
  The choice depends on the calculation.
\end{remark}

%==========================================================
\section{Cauchy--Schwarz Inequality (Inner Product Form)}
%==========================================================

\begin{theorem}[Cauchy--Schwarz Inequality]
  For any $\ket{u},\ket{v}$ in a Hilbert space $\mathcal{H}$:
  \[
    |\braket{u}{v}|^2 \leq \braket{u}{u}\cdot\braket{v}{v}.
  \]
  This bounds the squared inner product by the product of norms-squared.
\end{theorem}

\begin{remark}
  The Cauchy--Schwarz inequality:
  \begin{itemize}
    \item Ensures the inner product is well-defined on $\mathcal{H}$.
    \item Is equivalent to the statement $|\cos\theta|\leq 1$ in geometry.
    \item Is a cornerstone of functional analysis and quantum mechanics.
  \end{itemize}
\end{remark}
