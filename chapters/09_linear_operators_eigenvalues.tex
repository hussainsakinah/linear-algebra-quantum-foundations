\chapter{Linear Operators and Eigenvalues}

%==========================================================
\section{Linear Operators}
%==========================================================

In quantum mechanics:
\begin{itemize}
  \item \textbf{States} of a system are represented by vectors in a
        vector space (Hilbert space).
  \item \textbf{Physical observables} (energy, position, momentum, angular
        momentum) are represented by \emph{linear operators} on that space.
\end{itemize}

\begin{definition}[Linear Operator]
  A map $M: V(F) \to V(F)$ is a \emph{linear operator} if it satisfies
  (for all $\ket{A},\ket{B}\in V$ and $z\in F$):
  \begin{enumerate}[label=\textbf{LO\arabic*.}]
    \item $M\bigl(z\ket{A}\bigr) = z\,M\ket{A}$,
    \item $M\bigl(\ket{A}+\ket{B}\bigr) = M\ket{A}+M\ket{B}$.
  \end{enumerate}
  Think of $M$ as a machine: input goes in, output comes out.  If no output
  is produced, $M$ is not a linear operator.
\end{definition}

\begin{remark}
  Not every operator is linear.  Linear operators are \emph{associated with}
  a vector space but are not themselves vectors.
\end{remark}

%==========================================================
\section{Matrix Representation of Linear Operators}
%==========================================================

\begin{proposition}
  Given an orthonormal basis $\{|i\rangle\}$ of $V$, a linear operator $M$
  is represented by the matrix with entries
  \[
    m_{ki} = \langle k \mid M \mid i \rangle.
  \]
  The action $M\ket{A} = \ket{B}$ becomes
  \[
    \begin{pmatrix} m_{11} & m_{12} & m_{13} \\ m_{21} & m_{22} & m_{23} \\
    m_{31} & m_{32} & m_{33} \end{pmatrix}
    \begin{pmatrix} a_1 \\ a_2 \\ a_3 \end{pmatrix}
    = \begin{pmatrix} b_1 \\ b_2 \\ b_3 \end{pmatrix},
    \qquad
    \sum_i a_i\, m_{ki} = b_k.
  \]
\end{proposition}

\begin{remark}
  The dimensions of the matrix $M$ depend on the choice of basis.  The
  relationship between vectors and operators is \emph{independent} of basis
  in abstract notation but acquires a concrete matrix form once a basis is
  fixed.
\end{remark}

\paragraph{Special operators.}
\begin{itemize}
  \item \textbf{Identity operator:} $I_V\ket{v} = \ket{v}$,\quad $\ket{v}\in V(F)$.
  \item \textbf{Zero operator:} $O\ket{v} = 0$,\quad $\ket{v}\in V(F)$.
  \item \textbf{Composition:} If $V: A\to B$ and $W: B\to C$ are linear,
        then $WV\ket{a} = W(V\ket{a})$ is linear, and $AB(\ket{v}) = A(B\ket{v})$.
\end{itemize}

%==========================================================
\section{Eigenvalues and Eigenvectors}
%==========================================================

\begin{definition}[Eigenvalue / Eigenvector]
  A non-zero vector $\ket{\lambda}$ is an \emph{eigenvector} of $M$ with
  \emph{eigenvalue} $\lambda \in \C$ if
  \[
    \boxed{M\ket{\lambda} = \lambda\ket{\lambda}.}
  \]
  The operator $M$ preserves the \emph{direction} of $\ket{\lambda}$ (only
  scaling it by $\lambda$).
\end{definition}

\begin{remark}
  In general, a linear operator \emph{changes} the direction of its input
  vector.  Eigenvectors are special inputs for which only scaling occurs.
\end{remark}

\begin{example}
  Let $M = \begin{pmatrix}1&2\\2&1\end{pmatrix}$.
  \begin{enumerate}
    \item $\ket{v} = \begin{pmatrix}1\\1\end{pmatrix}$:
          $M\ket{v} = \begin{pmatrix}3\\3\end{pmatrix} = 3\ket{v}$.
          Eigenvalue $\lambda = 3$. \checkmark
    \item $\ket{u} = \begin{pmatrix}1\\-1\end{pmatrix}$:
          $M\ket{u} = \begin{pmatrix}-1\\1\end{pmatrix} = -1\ket{u}$.
          Eigenvalue $\lambda = -1$. \checkmark
    \item Let $N = \begin{pmatrix}0&-1\\1&0\end{pmatrix}$,
          $\ket{u'} = \begin{pmatrix}1\\i\end{pmatrix}$:
          $N\ket{u'} = \begin{pmatrix}-i\\1\end{pmatrix} = -i\ket{u'}$.
          Eigenvalue $\lambda = -i$. \checkmark
  \end{enumerate}
\end{example}

%==========================================================
\section{Hermitian Conjugate and Its Role in Quantum Mechanics}
%==========================================================

\begin{proposition}
  For a linear operator $M$, the map in dual space is given by:
  \[
    M\ket{A} = \ket{B} \quad\Longleftrightarrow\quad \bra{A}M^\dagger = \bra{B},
  \]
  so $M^\dagger$ is the Hermitian conjugate (adjoint) of $M$.  In matrix form,
  \[
    M = \begin{pmatrix}a&b\\c&d\end{pmatrix}
    \implies
    M^\dagger = \begin{pmatrix}a^*&c^*\\b^*&d^*\end{pmatrix}.
  \]
\end{proposition}

\begin{proof}[Derivation]
  Starting from $M\ket{A}=\ket{B}$, expand in orthonormal basis:
  $M\sum_i a_i\ket{i} = \sum_i b_i\ket{i}$.
  Apply $\bra{j}$: $\sum_i a_i \langle j|M|i\rangle = b_j$, so $m_{ji}$ are
  the matrix elements.  In dual space with complex-conjugate coefficients,
  $m_{ji}^*$ appear, giving $(M^\dagger)_{ij} = m_{ji}^* = (M^T)^*_{ij}$,
  i.e.\ $M^\dagger = (\bar M)^T$. \qed
\end{proof}

%==========================================================
\section{Observables and Hermitian Operators}
%==========================================================

\begin{theorem}[Observables are Hermitian]\label{thm:hermitian-real}
  Observables in quantum mechanics are represented by Hermitian operators
  ($M = M^\dagger$), and \emph{all eigenvalues of a Hermitian operator are
  real}.
\end{theorem}

\begin{proof}
  Let $M\ket{\lambda} = \lambda\ket{\lambda}$.  Since $M = M^\dagger$,
  \[
    \bra{\lambda}M^\dagger = \lambda^*\bra{\lambda}.
  \]
  Multiply $\bra{\lambda}$ on the left of $M\ket{\lambda}=\lambda\ket{\lambda}$:
  \[
    \braket{\lambda}{M|\lambda} = \lambda\braket{\lambda}{\lambda}.
  \]
  Multiply $M\ket{\lambda}=\lambda\ket{\lambda}$ on the right with $\bra{\lambda}$
  and use $M=M^\dagger$:
  \[
    \braket{\lambda}{M|\lambda} = \lambda^*\braket{\lambda}{\lambda}.
  \]
  Subtracting: $0 = (\lambda^*-\lambda)\braket{\lambda}{\lambda}$.
  Since $\ket{\lambda}\neq 0$ we have $\braket{\lambda}{\lambda}>0$, hence
  $\lambda^* = \lambda$, i.e.\ $\lambda \in \R$. \qed
\end{proof}

\begin{remark}
  This is physically essential: the outcome of any experiment is always a real
  number, and observables correspond to Hermitian operators precisely to
  guarantee this.
\end{remark}

%==========================================================
\section{Cauchy--Schwarz Inequality}
%==========================================================

\begin{theorem}[Cauchy--Schwarz Inequality]
  For any two vectors $\ket{u},\ket{v}$ in a Hilbert space $\mathcal{H}$,
  \[
    \abs{\braket{u}{v}}^2 \leq \braket{u}{u}\,\braket{v}{v}.
  \]
\end{theorem}

\begin{remark}
  The Cauchy--Schwarz inequality is fundamental to Hilbert space theory.  It
  bounds how much any two vectors can ``overlap'' and is a key ingredient in
  proving the triangle inequality for the induced norm.
\end{remark}
