\chapter{Fourier Analysis}

Fourier analysis is the mathematical heart of quantum mechanics.  The Fourier
transform is the mathematical heart of Heisenberg's uncertainty principle
(Rajan Chopra).  It decomposes functions into their constituent frequencies,
exactly as an orthonormal basis decomposes vectors.

%==========================================================
\section{Motivation: Seeing Functions Differently}
%==========================================================

The Fourier transform is a way of viewing a function from a different
perspective — the \emph{frequency domain} instead of the \emph{time domain}.
Just as $\hat{\imath},\hat{\jmath},\hat{k}$ span all of 3D space, the
exponential functions $e^{i n\omega_0 t}$ span $L^2$ space.

%==========================================================
\section{Fourier Series}
%==========================================================

\begin{definition}[Periodic Function]
  A function $f: \R \to \C$ is \emph{periodic} with period $T$ if
  $f(t) = f(t + T)$ for all $t$.
\end{definition}

\begin{theorem}[Fourier Series]
  Any periodic function $f(t)$ with period $T$ can be expressed as a sum of
  sines and cosines (each with its own amplitude and frequency):
  \[
    f(t) = a_0 + \sum_{n=1}^{\infty}
           \bigl[\, a_n \sin(n\omega_0 t) + b_n \cos(n\omega_0 t)\,\bigr],
  \]
  where $\omega_0 = 2\pi / T$ is the \emph{fundamental angular frequency} and
  $\omega = 2\pi\nu$ ($\nu$ = frequency in waves per second).
  Sines and cosines act as the \emph{basis functions}.
\end{theorem}

\begin{example}
  $f(t) = \sin\omega t + \tfrac{1}{2}\sin 2\omega t + \tfrac{3}{2}\cos\omega t$.
  Each term is one frequency component, so this function has been expressed in
  its ``Fourier basis.''
\end{example}

\subsection{Why Use Sines and Cosines?}

\begin{enumerate}
  \item \textbf{Orthogonality:} Sine and cosine are mutually independent
        (like orthogonal vectors in a vector space).
  \item \textbf{Completeness:} The set $\{\sin(n\omega_0 t), \cos(n\omega_0 t)\}$
        is a complete set — it can generate a wide variety of functions,
        especially from $L^2$.
\end{enumerate}

%==========================================================
\section{Complex Exponential Form}
%==========================================================

Using Euler's formula $e^{i\theta} = \cos\theta + i\sin\theta$:
\[
  \cos\theta = \frac{e^{i\theta}+e^{-i\theta}}{2}, \qquad
  \sin\theta = \frac{e^{i\theta}-e^{-i\theta}}{2i}.
\]

Substituting into the Fourier series and collecting terms, one obtains the
\textbf{complex exponential Fourier series}:

\begin{equation}\label{eq:complex-fourier}
  \boxed{f(t) = \sum_{n=-\infty}^{\infty} C_n\, e^{in\omega_0 t},}
\end{equation}

where $C_n \in \C$ are the \emph{complex Fourier coefficients}.

\paragraph{Why $e^{i\omega t}$?}  Three reasons:
\begin{enumerate}
  \item \textbf{Orthogonality:} $e^{i\omega t},\, e^{2i\omega t},\ldots$ are
        all mutually independent.
  \item \textbf{Completeness:} Any well-behaved function is a linear
        combination of $\{e^{in\omega_0 t}\}$.
  \item \textbf{Periodic nature:} $e^{i\omega t}$ traces a circle in the
        complex plane as $t$ varies.
\end{enumerate}

%==========================================================
\section{Derivation of Fourier Coefficients}
%==========================================================

\begin{theorem}[Fourier Coefficients]
  The $n$-th complex Fourier coefficient of a periodic function $f(t)$ with
  period $T$ is
  \[
    \boxed{C_n = \frac{1}{T}\int_0^T f(t)\, e^{-in\omega_0 t}\,dt.}
  \]
\end{theorem}

\begin{proof}
  Start from~\eqref{eq:complex-fourier}: $f(t) = \sum_{k=-\infty}^{\infty} C_k e^{ik\omega_0 t}$.

  \textbf{Step 1.} Multiply both sides by $e^{-in\omega_0 t}$ (for a fixed
  integer $n$) and integrate over one period $[0,T]$:
  \[
    \int_0^T e^{-in\omega_0 t} f(t)\,dt
    = \int_0^T e^{-in\omega_0 t} \sum_{k=-\infty}^{\infty} C_k e^{ik\omega_0 t}\,dt.
  \]

  \textbf{Step 2.} Exchange sum and integral (Fubini's theorem):
  \[
    = \sum_{k=-\infty}^{\infty} C_k \int_0^T e^{i(k-n)\omega_0 t}\,dt.
  \]

  \textbf{Step 3.} Use orthogonality of complex exponentials: any two
  exponentials $e^{ik\omega_0 t}$ and $e^{in\omega_0 t}$ are orthogonal over
  one period (their inner product is zero if $k\neq n$).  Explicitly, if
  $k \neq n$:
  \begin{align*}
    \int_0^T e^{i(k-n)\omega_0 t}\,dt
    &= \left[\frac{e^{i(k-n)\omega_0 t}}{i(k-n)\omega_0}\right]_0^T
     = \frac{e^{i(k-n)\cdot 2\pi}-1}{i(k-n)\omega_0} \\
    &= \frac{\cos 2\pi(k-n)+i\sin 2\pi(k-n)-1}{i(k-n)\omega_0}
     = \frac{1+0-1}{i(k-n)\omega_0} = 0.
  \end{align*}

  \textbf{Step 4.} Only the $k=n$ term survives:
  \[
    \int_0^T e^{-in\omega_0 t} f(t)\,dt = C_n \int_0^T 1\,dt = C_n T.
  \]

  \textbf{Step 5.} Solve for $C_n$:
  \[
    C_n = \frac{1}{T}\int_0^T f(t)\,e^{-in\omega_0 t}\,dt. \qed
  \]
\end{proof}

%==========================================================
\section{Fourier Transform}
%==========================================================

\begin{definition}[Fourier Transform]
  For a \emph{non-periodic} function $f(t)$ (thought of as having a
  continuous spectrum of frequencies), the \textbf{Fourier transform} is
  \[
    \boxed{\tilde{f}(\omega) = \int_{-\infty}^{\infty} f(t)\, e^{-i\omega t}\,dt.}
  \]
  This transforms from the \emph{time domain} to the \emph{frequency domain}.
  $\tilde{f}(\omega)$ is the \emph{amplitude} of the frequency component
  $\omega$ in the original function — it tells us how much $e^{i\omega t}$
  contributes to $f(t)$.
\end{definition}

\begin{definition}[Inverse Fourier Transform]
  The original function is recovered by
  \[
    \boxed{f(t) = \frac{1}{2\pi}\int_{-\infty}^{\infty} \tilde{f}(\omega)\, e^{i\omega t}\,d\omega.}
  \]
  This transforms from the \emph{frequency domain} back to the \emph{time
  domain}.  The factor $\frac{1}{2\pi}$ is the normalizing factor.
\end{definition}

\begin{remark}
  A non-periodic function can be thought of as having a \emph{continuous}
  spectrum of frequencies.  It can be represented as a continuous sum
  (integral) of sines, cosines, or exponentials $e^{i\omega t}$,
  over (possibly) all frequencies.
\end{remark}

\begin{example}[Gaussian function]
  Let $f(t) = e^{-t^2}$.  Then
  \[
    \tilde{f}(\omega) = \int_{-\infty}^{\infty} e^{-t^2} e^{-i\omega t}\,dt
                      = \frac{1}{\sqrt{\pi}}\,e^{-\omega^2/4},
  \]
  so
  \[
    e^{-t^2} = \int_{-\infty}^{\infty}
               \underbrace{\frac{1}{\sqrt{\pi}} e^{-\omega^2/4}}_{\text{coeff.}}
               \underbrace{e^{i\omega t}}_{\text{basis}}\,d\omega.
  \]
  A Gaussian in time is a Gaussian in frequency — a hallmark of the
  uncertainty principle.
\end{example}

%==========================================================
\section{Connection to Quantum Mechanics}
%==========================================================

The Fourier transform is the heart of \textbf{Heisenberg's Uncertainty
Principle}.

\begin{itemize}
  \item A sharply localized (narrow) wavepacket in position-space
        corresponds to a \emph{spread-out} wavefunction in momentum-space,
        and vice versa.
  \item Mathematically: $\Delta x\, \Delta p \geq \hbar/2$.
  \item The Fourier transform relates the two representations.
\end{itemize}

The analogy with linear algebra is exact:
\[
  \underbrace{\ket{v} = \sum_i c_i\ket{e_i}}_{\text{vector expansion}}
  \quad\longleftrightarrow\quad
  \underbrace{f(t) = \int \tilde{f}(\omega)\,e^{i\omega t}\,d\omega}_{\text{Fourier expansion}},
\]
with $e^{i\omega t}$ playing the role of the orthonormal basis functions and
$\tilde{f}(\omega)$ playing the role of the expansion coefficients $c_i$.
