\chapter{Complex Numbers}

Complex numbers are the native number system of quantum mechanics.  Every
quantum amplitude is complex, and the modulus-squared gives a probability.

%==========================================================
\section{Definition and the Argand Plane}
%==========================================================

\begin{definition}[Imaginary Unit]
  The \emph{imaginary unit} is the number $i$ satisfying $i^2 = -1$, i.e.\
  $i = \sqrt{-1}$.
\end{definition}

\begin{definition}[Complex Number]
  A \emph{complex number} is an expression of the form
  \[
    z = x + iy, \qquad x,y \in \R,
  \]
  where $x = \operatorname{Re}(z)$ is the \emph{real part} and
  $y = \operatorname{Im}(z)$ is the \emph{imaginary part}.
\end{definition}

The set $\C$ of all complex numbers is visualised on the \textbf{Argand plane}
(complex plane), with the real axis horizontal and the imaginary axis vertical.

%==========================================================
\section{Polar Form and Euler's Formula}
%==========================================================

\begin{definition}[Polar Form]
  Any $z = x+iy \in \C$ can be written in \emph{polar form}
  \[
    z = r(\cos\theta + i\sin\theta),
  \]
  where $r = |z| = \sqrt{x^2+y^2}$ is the \emph{modulus} and
  $\theta = \arg(z)$ is the \emph{argument}.
\end{definition}

\begin{theorem}[Euler's Formula]
  \[
    e^{i\theta} = \cos\theta + i\sin\theta.
  \]
  Hence the polar form simplifies to $z = r e^{i\theta}$.
\end{theorem}

\begin{corollary}[Euler's Identity]
  Setting $\theta = \pi$:
  \[
    \boxed{e^{i\pi} + 1 = 0.}
  \]
\end{corollary}

\begin{proposition}[Product of complex numbers]
  If $z_1 = r_1 e^{i\theta_1}$ and $z_2 = r_2 e^{i\theta_2}$, then
  \[
    z_1 z_2 = r_1 r_2\, e^{i(\theta_1+\theta_2)}.
  \]
  Multiplication \emph{adds arguments} and \emph{multiplies moduli}.
\end{proposition}

%==========================================================
\section{Complex Conjugate}
%==========================================================

\begin{definition}[Complex Conjugate]
  The \emph{complex conjugate} of $z = x+iy$ is $z^* = x - iy$.
  Geometrically, $z^*$ is the reflection of $z$ through the real axis
  (angle $-\theta$).
\end{definition}

\begin{proposition}[Modulus via conjugate]
  \[
    z z^* = (x+iy)(x-iy) = x^2 + y^2 = |z|^2.
  \]
  Hence $\abs{z} = \sqrt{z z^*}$.
\end{proposition}

\begin{proof}
  Using Euler's form:
  \[
    z z^* = (re^{i\theta})(re^{-i\theta}) = r^2 e^0 = r^2. \qed
  \]
\end{proof}

\begin{remark}[Dual number system]
  The pair $(z, z^*)$ forms a \emph{dual number system}: for every $z \in \C$
  there exists a unique $z^*$, so that $\forall z\; \exists!\, z^*$.
\end{remark}

%==========================================================
\section{Phase Factor}
%==========================================================

\begin{definition}[Phase Factor]
  A \emph{phase factor} is a complex number of unit modulus:
  \[
    z = e^{i\theta} = \cos\theta + i\sin\theta, \qquad |z| = 1.
  \]
  Phase factors are ubiquitous in quantum mechanics: quantum states often
  differ only by a phase, and this phase carries physical information (e.g.\
  interference).
\end{definition}

\begin{remark}
  In quantum calculations we work freely in $\C$, but the \emph{final answer
  must be real} — probabilities and expectation values of Hermitian observables
  are always real numbers.
\end{remark}

%==========================================================
\section{Complex Vector Space}
%==========================================================

\begin{definition}[Complex Vector Space]
  A \emph{complex vector space} $V(\C)$ is a vector space over the field
  $\C$ of scalars.  This is the setting for quantum mechanics, where we take
  $F = \C$.
\end{definition}

The complex conjugate of the scalar field $\C$ is $\C^*$; the dual vector
space $V^*(\C^*)$ is the \emph{complex conjugate vector space}.
