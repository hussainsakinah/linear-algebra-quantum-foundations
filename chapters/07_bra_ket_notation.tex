\chapter{Bra-Ket (Dirac) Notation}

Dirac's bra-ket notation is the standard language of quantum mechanics.  It
provides an elegant, basis-independent way to express vectors, dual vectors,
inner products, and operators.

%==========================================================
\section{Notation Table}
%==========================================================

\begin{center}
\renewcommand{\arraystretch}{1.4}
\begin{tabular}{@{}lll@{}}
  \toprule
  Symbol & Name & Meaning \\
  \midrule
  $z$ & Complex scalar & $z \in \C$ \\
  $z^*$ & Complex conjugate & $z^* \in \C^*$ \\
  $\ket{\psi}$ & Ket vector & $\ket{\psi} \in V(\C)$, the complex vector space \\
  $\bra{\psi}$ & Bra vector & $\bra{\psi} \in V^*(\C^*)$, the dual (conjugate) space \\
  $\braket{\phi}{\psi}$ & Inner product & A number (generally $\in\C$) \\
  $\ket{\phi}\otimes\ket{\psi}$ & Tensor product & A vector in $V\otimes V$ \\
  $A^*$ & Complex conjugate of matrix $A$ & Element-wise conjugation \\
  $A^T$ & Transpose & $(A^T)_{ij} = A_{ji}$ \\
  $A^\dagger$ & Hermitian conjugate & $A^\dagger = (A^T)^* = (\bar{A})^T$ \\
  \bottomrule
\end{tabular}
\end{center}

%==========================================================
\section{Ket Vectors — Axioms}
%==========================================================

Let $V(\C)$ be a complex vector space.  A ket $\ket{\psi}\in V(\C)$ represents
the \emph{state of a quantum particle}.  Ket vectors satisfy the following
seven axioms (for $\ket{\psi},\ket{\phi},\ket{\chi}\in V$ and $z,w\in\C$):

\begin{enumerate}[label=\textbf{K\arabic*.}]
  \item $\ket{\psi}+\ket{\phi} \in V$ \hfill(closure under $+$)
  \item $\ket{\psi}+\ket{\phi} = \ket{\phi}+\ket{\psi}$ \hfill(commutativity)
  \item $\ket{\psi}+(\ket{\phi}+\ket{\chi}) = (\ket{\psi}+\ket{\phi})+\ket{\chi}$ \hfill(associativity)
  \item $\ket{\psi}+\ket{0} = \ket{\psi}$ \hfill(additive identity)
  \item $\ket{\psi}+(-\ket{\psi}) = \ket{0}$ \hfill(additive inverse)
  \item $z\ket{\psi} = z\ket{\psi} \in V$ \hfill(scalar multiplication by $\C$)
  \item $z(\ket{A}+\ket{B}) = z\ket{A}+z\ket{B}$, \quad
        $(z+w)\ket{A} = z\ket{A}+w\ket{A}$ \hfill(linearity)
\end{enumerate}

A function $A(x)$ that is continuous and complex-valued, closed under addition
and complex-scalar multiplication, and satisfies the above axioms is a ket
vector.

\begin{example}[Column vector ket]
  \[
    \ket{\psi} = \begin{pmatrix}\alpha_1\\\alpha_2\\\vdots\\\alpha_n\end{pmatrix}, \qquad
    \ket{\psi}+\ket{\phi} = \begin{pmatrix}\alpha_1+\beta_1\\\alpha_2+\beta_2\end{pmatrix}, \qquad
    z\ket{\psi} = \begin{pmatrix}z\alpha_1\\z\alpha_2\end{pmatrix}.
  \]
\end{example}

%==========================================================
\section{Bra Vectors (Dual Space)}
%==========================================================

\begin{definition}[Bra Vector]
  For every ket $\ket{a} \in V$ there exists a unique \emph{bra}
  $\bra{a} \in V^*$ in the dual space (complex conjugate vector space):
  \[
    \ket{a} = \begin{pmatrix}\alpha_1\\\alpha_2\\\vdots\\\alpha_n\end{pmatrix}
    \quad\longleftrightarrow\quad
    \bra{a} = \bigl(\alpha_1^*\; \alpha_2^*\; \cdots\; \alpha_n^*\bigr).
  \]
\end{definition}

\begin{proposition}[Duality under scaling]
  If $z\ket{A} = \ket{B}$, then $\bra{A}z^* = \bra{B}$, i.e.,
  $z\ket{A} \Rightarrow \bra{A}z^*$.
\end{proposition}

\begin{remark}
  Bra vectors also satisfy the seven axioms above (they live in the dual
  complex-conjugate space $V^*(\C^*)$), and
  $\forall\, \ket{a}\in V\; \exists!\, \bra{a} \in V^*$.
\end{remark}

%==========================================================
\section{Inner Product in Bra-Ket Notation}
%==========================================================

\begin{definition}
  The inner product of $\ket{\phi}$ and $\ket{\psi}$ is
  \[
    \braket{\phi}{\psi} \in \C.
  \]
  Properties:
  \begin{enumerate}
    \item \textbf{Linearity:}
          $\bra{\psi}(\ket{\phi}+\ket{\chi}) = \braket{\psi}{\phi}+\braket{\psi}{\chi}$.
    \item \textbf{Conjugate property:}
          $\braket{A}{B} = \braket{B}{A}^*$.
    \item \textbf{Reality of self-inner-product:}
          $\braket{A}{A} \in \R_{\geq 0}$ (with some exceptions
          involving complex phases).
  \end{enumerate}
\end{definition}

%==========================================================
\section{Length of a Vector in Bra-Ket Notation}
%==========================================================

For $z = x+iy$,
\[
  zz^* = (x+iy)(x-iy) = x^2 + y^2,
\]
so the norm of $\ket{\psi}=(\alpha_1,\alpha_2,\alpha_3)^T$ is
\[
  \|\ket{\psi}\| = \sqrt{\alpha_1^2 + \alpha_2^2 + \alpha_3^2}.
\]
A vector of length~1 is a \textbf{unit vector}.

\begin{example}[Standard basis kets for a qubit]
  \begin{alignat*}{2}
    \ket{\uparrow} &= \begin{pmatrix}1\\0\end{pmatrix}, \quad &
    \ket{\downarrow} &= \begin{pmatrix}0\\1\end{pmatrix}, \\[6pt]
    \ket{\rightarrow} &= \frac{1}{\sqrt{2}}\begin{pmatrix}1\\-1\end{pmatrix}, \quad &
    \ket{\leftarrow} &= \frac{1}{\sqrt{2}}\begin{pmatrix}1\\1\end{pmatrix}, \\[6pt]
    \ket{\nearrow} &= \begin{pmatrix}1/2\\-\sqrt{3}/2\end{pmatrix}, \quad &
    \ket{\swarrow} &= \begin{pmatrix}\sqrt{3}/2\\1/2\end{pmatrix}.
  \end{alignat*}
  These all have unit norm. The standard basis $\{\ket{\uparrow},\ket{\downarrow}\}$
  is used for the spin of an electron.
\end{example}

%==========================================================
\section{Basis, Linear Combination, and Linear Dependence}
%==========================================================

\begin{definition}[Basis]
  A set $\{|i\rangle\}$ of vectors in $V(F)$ is a \emph{basis} if it
  \begin{enumerate}
    \item \textbf{Spans} $V$ (every vector in $V$ is a linear combination
          of basis vectors), and
    \item is \textbf{linearly independent}.
  \end{enumerate}
  An \emph{orthonormal basis} additionally satisfies $\langle i|j\rangle =
  \delta_{ij}$.
\end{definition}

\begin{theorem}[Expansion in an orthonormal basis]
  If $\{|i\rangle\}$ is an orthonormal basis of $V(F)$, then any
  $\ket{A}\in V$ can be written
  \[
    \ket{A} = \sum_i \braket{i}{A}\,\ket{i},
  \]
  where the coefficient $\braket{i}{A}$ is the \emph{component} of $\ket{A}$
  along $\ket{i}$.
\end{theorem}

\begin{proof}
  Write $\ket{A} = \sum_i \alpha_i\ket{i}$.  Apply $\bra{j}$ to both sides:
  \[
    \braket{j}{A} = \sum_i \alpha_i \braket{j}{i} = \sum_i \alpha_i\,\delta_{ji} = \alpha_j. \qed
  \]
\end{proof}

\begin{definition}[Linear Dependence and Independence]
  A set of vectors $\{\ket{v_1},\ldots,\ket{v_n}\}$ is
  \emph{linearly dependent} if there exist scalars $c_1,\ldots,c_n$, not all
  zero, such that
  \[
    c_1\ket{v_1} + c_2\ket{v_2} + \cdots + c_n\ket{v_n} = 0.
  \]
  If the only solution is $c_1=c_2=\cdots=c_n=0$, the set is
  \emph{linearly independent}.
\end{definition}

\begin{proposition}[Dimension]
  Any two linearly independent sets that span $V$ have the same cardinality,
  which defines the \emph{dimension} of $V$.  In quantum computing, Hilbert
  spaces are finite-dimensional; in quantum mechanics, infinite-dimensional.
\end{proposition}

%==========================================================
\section{Probability Amplitudes}
%==========================================================

For an infinite-dimensional expansion
\[
  \ket{\Psi} = \sum_n a_n\ket{b_n},
\]
the \emph{probability amplitude} that $\ket{\Psi}$ ``jumps'' to state
$\ket{b_k}$ upon measurement is $\braket{b_k}{\Psi}$, and the probability
is
\[
  P(\text{outcome }k) = |\braket{b_k}{\Psi}|^2 = \braket{b_k}{\Psi}\braket{\Psi}{b_k}.
\]
