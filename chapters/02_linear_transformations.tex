\chapter{Linear Transformations and Systems of Equations}

%==========================================================
\section{Systems of Linear Equations}
%==========================================================

A system of linear equations can have three qualitatively different solution
sets, visualised geometrically as intersecting lines in $\R^2$:

\begin{enumerate}
  \item \textbf{Unique solution} — lines intersect at exactly one point.
  \item \textbf{No solution} — lines are parallel and distinct.
  \item \textbf{Infinitely many solutions} — lines coincide.
\end{enumerate}

\begin{example}
  \[
    \begin{cases} x + y = 6 \\ 2x - y = 3 \end{cases}
    \xrightarrow{\text{subtract}}
    -x = 3 \implies x = -3,\; y = 9.
  \]
  Unique solution. \qed
\end{example}

%==========================================================
\section{Matrix Notation (Cayley)}
%==========================================================

A system of $m$ equations in $n$ unknowns is written compactly as
\[
  A\mathbf{x} = \mathbf{b},
\]
where $A$ is an $m\times n$ matrix, $\mathbf{x}$ is the $n\times 1$ column of
unknowns, and $\mathbf{b}$ is the $m\times 1$ right-hand-side vector.

\begin{example}
  \[
    \begin{pmatrix} 1 & 1 \\ 2 & -1 \end{pmatrix}
    \begin{pmatrix} x \\ y \end{pmatrix}
    = \begin{pmatrix} 6 \\ 4 \end{pmatrix},
    \qquad
    \begin{pmatrix} 1 & 2 & 1 \\ 1 & -3 & 1 \\ 2 & -1 & 1 \end{pmatrix}
    \begin{pmatrix} x \\ y \\ z \end{pmatrix}
    = \begin{pmatrix} 0 \\ 3 \\ 5 \end{pmatrix}.
  \]
  The matrix is \emph{responsible for the transformation} of the solution
  space.
\end{example}

%==========================================================
\section{Linear Transformations}
%==========================================================

\begin{definition}[Linear Transformation]
  Let $V$ and $W$ be vector spaces over the same field $F$.  A map
  $T : V \to W$ is a \emph{linear transformation} if it preserves the vector
  space structure:
  \begin{align}
    T(\mathbf{u} + \mathbf{v}) &= T(\mathbf{u}) + T(\mathbf{v})
      && \forall\, \mathbf{u},\mathbf{v}\in V, \label{eq:lt-add}\\
    T(\gamma\, \mathbf{u}) &= \gamma\, T(\mathbf{u})
      && \forall\, \mathbf{u}\in V,\; \gamma \in F. \label{eq:lt-scalar}
  \end{align}
\end{definition}

\begin{example}[Geometric transformations on $\R^2$]
  Define
  \[
    T(x,y) = (x, y), \qquad T(x,y) = (x+y,\; 2x-y).
  \]
  The second map sends integer-lattice points to a \emph{tilted} lattice —
  a square grid becomes a tilted rectangle.  This geometric distortion is the
  hallmark of a non-trivial linear transformation.
\end{example}

\subsection{Reflections as Linear Transformations}

\begin{example}
  Define the two maps $G,H: \R^2 \to \R^2$ by
  \[
    G(x,y) = (x,\,-y), \qquad H(x,y) = (y,\,-x).
  \]
  Their matrix representations are
  \[
    G = \begin{pmatrix} 1 & 0 \\ 0 & -1 \end{pmatrix}, \qquad
    H = \begin{pmatrix} 0 & -1 \\ 1 & 0 \end{pmatrix}.
  \]
  The composition $GH$ satisfies
  \begin{align*}
    (GH)(x,y)
      &= G(H(x,y)) = G(y,-x) = (y,x),
  \end{align*}
  so
  \[
    GH = \begin{pmatrix} 1 & 0 \\ 0 & -1 \end{pmatrix}
         \begin{pmatrix} 0 & -1 \\ 1 & 0 \end{pmatrix}
       = \begin{pmatrix} 0 & 1 \\ 1 & 0 \end{pmatrix},
    \qquad
    \begin{pmatrix} 0 & 1 \\ 1 & 0 \end{pmatrix}
    \begin{pmatrix} x \\ y \end{pmatrix}
    = \begin{pmatrix} y \\ x \end{pmatrix}.
  \]
  \textbf{Verification:} $GH(1,2)$: first $H(1,2)=(2,-1)$, then $G(2,-1)=(2,1)$.
  \qed
\end{example}

%==========================================================
\section{Composition of Linear Transformations as Matrix Multiplication}
%==========================================================

\begin{proposition}[Matrix multiplication encodes composition]
  If $T_1 : \R^2 \to \R^2$ has matrix $\begin{psmallmatrix} a & b \\ c & d \end{psmallmatrix}$
  and $T_2$ has matrix $\begin{psmallmatrix} A & B \\ C & D \end{psmallmatrix}$,
  then $(T_2 \circ T_1)$ has matrix
  \[
    \begin{pmatrix} A & B \\ C & D \end{pmatrix}
    \begin{pmatrix} a & b \\ c & d \end{pmatrix}
    = \begin{pmatrix} Aa+Bc & Ab+Bd \\ Ca+Dc & Cb+Dd \end{pmatrix}.
  \]
\end{proposition}

\begin{proof}
  Let $x' = Ax + By$, $y' = Cx + Dy$ and $x = ax_0 + by_0$, $y = cx_0 + dy_0$.
  Substituting:
  \begin{align*}
    x' &= A(ax_0+by_0) + B(cx_0+dy_0) = (Aa+Bc)x_0 + (Ab+Bd)y_0, \\
    y' &= C(ax_0+by_0) + D(cx_0+dy_0) = (Ca+Dc)x_0 + (Cb+Dd)y_0. \qed
  \end{align*}
\end{proof}

%==========================================================
\section{Abstract Formalism: Linear Transformations on $\R^2$}
%==========================================================

Let $(a,b),(c,d) \in \R^2$ (a 2-dimensional plane, i.e.\ a \emph{space}).
The scalar field $\gamma \in \R$ (or $\C$) specifies the number of components.
A map $T : \R^2 \to \R^2$ is linear if and only if

\begin{equation}\label{eq:lin-formal}
  T\bigl[(a,b)+(c,d)\bigr] = T(a,b)+T(c,d), \qquad
  T\bigl(\gamma(a,b)\bigr) = \gamma\, T(a,b).
\end{equation}

\begin{remark}
  A vector space with \emph{infinitely many dimensions} is called a
  \textbf{Hilbert Space} (see Chapter~\ref{chap:hilbert}).
\end{remark}
