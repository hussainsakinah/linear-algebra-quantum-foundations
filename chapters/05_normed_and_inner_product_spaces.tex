\chapter{Normed Spaces and Inner Product Spaces}

The progression \textbf{Vector Space} $\subset$ \textbf{Normed Space}
$\subset$ \textbf{Inner Product Space} $\subset$ \textbf{Hilbert Space}
adds richer geometric structure at each step.

%==========================================================
\section{Normed Spaces}
%==========================================================

A vector space has no built-in notion of length or distance.  A norm supplies
both.

\begin{definition}[Norm]
  Let $V$ be a vector space over $F \in \{\R, \C\}$.  A \emph{norm} on $V$
  is a function $\norm{\cdot}: V \to \R_{\geq 0}$ satisfying, for all
  $\mathbf{u},\mathbf{v}\in V$ and $a\in F$:
  \begin{enumerate}[label=\textbf{N\arabic*.}]
    \item \textbf{Non-negativity:} $\norm{\mathbf{u}} \geq 0$, and
          $\norm{\mathbf{u}}=0 \iff \mathbf{u}=\mathbf{0}$.
    \item \textbf{Homogeneity:} $\norm{a\mathbf{u}} = |a|\,\norm{\mathbf{u}}$.
    \item \textbf{Triangle inequality:} $\norm{\mathbf{u}+\mathbf{v}}
          \leq \norm{\mathbf{u}} + \norm{\mathbf{v}}$.
  \end{enumerate}
  The pair $(V, \norm{\cdot})$ is a \emph{normed space}.
\end{definition}

\begin{remark}
  The norm assigns a non-negative real number to each vector (its
  ``length''), but without an inherent notion of actual physical distance or
  angle between vectors.  The norm also induces a metric:
  \[
    d(\mathbf{u},\mathbf{v}) \coloneqq \norm{\mathbf{u}-\mathbf{v}}.
  \]
\end{remark}

%==========================================================
\section{Inner Product Spaces}
%==========================================================

\begin{definition}[Inner Product]\label{def:inner-product}
  Let $V$ be a vector space over $F$.  An \emph{inner product} is a map
  $\langle\cdot,\cdot\rangle: V\times V \to F$ (producing only
  non-negative real outputs when both arguments coincide) satisfying:
  \begin{enumerate}[label=\textbf{IP\arabic*.}]
    \item \textbf{Linearity in the first argument:}
          $\langle ax_1 + bx_2,\, y \rangle = a\langle x_1, y\rangle
           + b\langle x_2, y\rangle$.
    \item \textbf{Conjugate symmetry:}
          $\langle x, y\rangle = \overline{\langle y, x\rangle}$.
    \item \textbf{Positive definiteness:}
          $\langle x,x\rangle \geq 0$, and $\langle x,x\rangle = 0 \iff x = 0$.
  \end{enumerate}
  The pair $(V,\langle\cdot,\cdot\rangle)$ is an \emph{inner product space}
  (IPS).
\end{definition}

\begin{remark}
  Geometrically, the inner product $\langle x,y\rangle$ encodes how much $x$
  ``points in the direction of'' $y$ and how their magnitudes contribute to
  their alignment.  In physics it is often written
  $\vec{A}\cdot\vec{B} = |\vec{A}||\vec{B}|\cos\theta$.
\end{remark}

\subsection{Norm Induced by the Inner Product}

\begin{proposition}
  Every inner product induces a norm by
  \[
    \norm{x} \coloneqq \sqrt{\langle x, x\rangle}.
  \]
\end{proposition}

\begin{example}
  For $\vec{A}$ pointing along itself,
  $\vec{A}\cdot\vec{A} = |\vec{A}|^2\cos 0 = |\vec{A}|^2$,
  so $|\vec{A}| = \sqrt{\vec{A}\cdot\vec{A}}$.
\end{example}

%==========================================================
\section{Orthogonality, Normalization, and Orthonormality}
%==========================================================

\begin{definition}[Orthogonality]
  Two vectors $\ket{A}$ and $\ket{B}$ in an IPS are \emph{orthogonal} if
  $\braket{A}{B} = 0$.  Orthogonal vectors are linearly independent.
\end{definition}

\begin{definition}[Normalization]
  A vector $\ket{A}$ is \emph{normalized} (a unit vector) if
  $\braket{A}{A} = 1$.
\end{definition}

\begin{definition}[Orthonormality]
  A set of vectors $\{\ket{A}, \ket{B}, \ldots\}$ is \emph{orthonormal} if
  \[
    \braket{A}{B} = \delta_{AB} =
    \begin{cases} 0 & A \neq B, \\ 1 & A = B. \end{cases}
  \]
\end{definition}

%==========================================================
\section{Topological Properties of Spaces}
%==========================================================

\begin{definition}[Topological Space]
  A \emph{topological space} is the most general type of space: a set $X$
  together with a \emph{topology} (a collection $\mathcal{T}$ of
  ``open'' subsets of $X$) satisfying:
  \begin{enumerate}
    \item The union of \emph{any} collection of open sets is open.
    \item The intersection of a \emph{finite} number of open sets is open.
  \end{enumerate}
  Examples: function spaces, polynomial spaces.
\end{definition}

Three key properties of topological spaces relevant to quantum mechanics:

\begin{definition}[Connectedness]
  A space is \emph{connected} if it cannot be split into two
  non-overlapping, disjoint, non-empty open sets.
\end{definition}

\begin{definition}[Continuity]
  A function $f$ between topological spaces is \emph{continuous} (smooth) if
  small changes in the input produce only small changes in the output.
\end{definition}

\begin{definition}[Compactness]
  A space is \emph{compact} if it is both bounded and closed, making it
  well-behaved for convergence and containment.  For example, the open
  interval $(a,b)$ is \emph{not} compact; the closed interval $[a,b]$
  \emph{is} compact.
\end{definition}

%==========================================================
\section{Unit Vectors and Normalization}
%==========================================================

\begin{definition}[Unit Vector]
  A vector $\ket{\psi}$ with $\norm{\ket{\psi}} = 1$ is a \emph{unit vector}.
\end{definition}

\begin{proposition}[Normalization]
  Any non-zero vector $\ket{\psi}$ can be converted to a unit vector by
  dividing by its norm:
  \[
    \widehat{\psi} = \frac{1}{\norm{\ket{\psi}}}\ket{\psi}.
  \]
\end{proposition}

\begin{example}
  Let $\ket{\psi} = \begin{pmatrix}3\\1\end{pmatrix}$.
  Then $\norm{\ket{\psi}} = \sqrt{9+1} = \sqrt{10}$, and
  \[
    \text{unit vector} = \frac{1}{\sqrt{10}}\begin{pmatrix}3\\1\end{pmatrix}
    = \begin{pmatrix}3/\sqrt{10}\\1/\sqrt{10}\end{pmatrix},
    \qquad
    \left\|\begin{pmatrix}3/\sqrt{10}\\1/\sqrt{10}\end{pmatrix}\right\|
    = \sqrt{\tfrac{9}{10}+\tfrac{1}{10}} = 1. \quad\checkmark
  \]
\end{example}
