\chapter{Hilbert Spaces}
\label{chap:hilbert}

A \emph{Hilbert space} is the mathematical arena of quantum mechanics.  It
combines the algebraic richness of an inner product space with the analytical
property of completeness.

%==========================================================
\section{Cauchy Sequences and Convergence}
%==========================================================

\begin{definition}[Cauchy Sequence]
  A sequence $\{x_n\}$ in a metric space $(X,d)$ (or normed space) is a
  \emph{Cauchy sequence} if
  \[
    \forall\, \varepsilon > 0\;\; \exists\, N \in \N \;\text{ such that }\;
    \forall\, m,n \geq N: \quad |x_m - x_n| < \varepsilon.
  \]
  Intuitively, consecutive terms cluster arbitrarily tightly as $n\to\infty$.
\end{definition}

\begin{example}
  $\left\{\dfrac{1}{n}\right\}_{n=1}^{\infty} = 1,\tfrac{1}{2},\tfrac{1}{3},\ldots$
  is Cauchy: for any $\varepsilon>0$ choose $N > 2/\varepsilon$, then for
  $m,n\geq N$,
  \[
    |x_m-x_n| \leq |x_m| + |x_n| < \tfrac{2}{N} < \varepsilon.
  \]
\end{example}

\begin{definition}[Convergent Sequence]
  A sequence $\{x_n\}$ \emph{converges} to a limit $L$ if
  \[
    \lim_{n\to\infty} x_n = L, \quad\text{i.e.,}\quad
    \forall\,\varepsilon>0\;\exists\,N:\; n\geq N \Rightarrow |x_n - L| < \varepsilon.
  \]
  Every convergent sequence is Cauchy, but the converse requires completeness.
\end{definition}

\begin{definition}[Dart-board analogy]
  Think of convergence as a dart game: every throw lands strictly closer to
  the bull's-eye than the previous throw.  The sequence of positions is Cauchy;
  the bull's-eye is the limit.
\end{definition}

%==========================================================
\section{Completeness}
%==========================================================

\begin{definition}[Complete Normed Space (Banach Space)]
  A normed space $X$ is \emph{complete} if every Cauchy sequence in $X$
  has a limit \emph{within $X$}.  A complete normed space is called a
  \textbf{Banach space}.
\end{definition}

\begin{example}[Incompleteness of $\Q$]
  Consider the sequence in $\Q$ defined by $x_n = \sum_{k=0}^{n}\dfrac{1}{k!}$.
  It is Cauchy, but its limit $e = 2.71828\ldots$ is irrational, hence
  $e \notin \Q$.  Therefore $(\Q, |\cdot|)$ is \emph{not} complete.
\end{example}

\begin{example}[Motivation from physics]
  The sequence of hydrogen energy levels $E_n = -13.6\,\text{eV}/n^2$.
  As $n\to\infty$, $E_n \to 0$.  Completeness requires that $0$ be in the
  same space as the $E_n$, which it is (the continuum of unbound states).
\end{example}

%==========================================================
\section{Hilbert Space}
%==========================================================

\begin{definition}[Hilbert Space]
  A \textbf{Hilbert space} $\mathcal{H}$ is a \emph{complete inner product
  space} (a complete IPS):
  \[
    \mathcal{H} = \underbrace{\text{IPS}}_{\text{properties}}
                  + \underbrace{\text{completeness}}_{\text{every Cauchy seq.\ converges in }\mathcal{H}}.
  \]
\end{definition}

\subsection{Quantum Mechanical Interpretation}

\begin{enumerate}
  \item \textbf{States.} The state of a quantum particle is represented by a
        vector (ket) $\ket{\psi} \in \mathcal{H}$.
  \item \textbf{Infinite dimension.} $\mathcal{H}$ is generically
        infinite-dimensional, encoding the infinite complexity of states
        available to a particle.  A particle's state is a superposition of
        infinitely many basis states.
  \item \textbf{Finite vs.\ infinite.} In quantum \emph{computing} (qubits,
        spin), finite-dimensional Hilbert spaces suffice (2 or 3 dimensions).
        In full quantum \emph{mechanics} (position, momentum, energy), we
        need infinite-dimensional $\mathcal{H}$.
  \item \textbf{Many particles.} For a system of many particles, the total
        Hilbert space is the tensor product of individual Hilbert spaces:
        \[
          \mathcal{H}_{\text{total}} = \mathcal{H}_1 \otimes \mathcal{H}_2
                                        \otimes \cdots
        \]
        The space grows very large, but each individual factor remains
        infinite-dimensional.
  \item \textbf{Physical examples.}
        \begin{itemize}
          \item Spin of an electron, polarisation of a photon
                $\longrightarrow$ 2 or 3 dimensions.
          \item Position, energy, momentum
                $\longrightarrow$ $\infty$ dimensions.
        \end{itemize}
\end{enumerate}

%==========================================================
\section{$L^2$ Space}
%==========================================================

\begin{definition}[$L^2$ Space]
  The space of \emph{square-integrable functions} is
  \[
    L^2(\R) = \left\{\, f : \R \to \C \;\middle|\;
               \int_{-\infty}^{\infty} |f(x)|^2\,dx < \infty \,\right\}.
  \]
  This is an infinite-dimensional Hilbert space.  Functions in $L^2$ are the
  natural setting for quantum wavefunctions.
\end{definition}

\begin{remark}
  In $L^2$, the vector $\ket{v} = c_1\ket{e_1} + c_2\ket{e_2} + \cdots$
  has basis elements $\ket{e_i}$ and coefficients $c_i$ that tell how much
  each basis element contributes to reconstruct the original ``vector''
  (function).  This is directly analogous to Fourier series coefficients.
\end{remark}
